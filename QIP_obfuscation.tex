\documentclass[11pt]{amsart}

%    Q-circuit version 2
%    Copyright (C) 2004  Steve Flammia & Bryan Eastin
%    Last modified on: 9/16/2011
%
%    This program is free software; you can redistribute it and/or modify
%    it under the terms of the GNU General Public License as published by
%    the Free Software Foundation; either version 2 of the License, or
%    (at your option) any later version.
%
%    This program is distributed in the hope that it will be useful,
%    but WITHOUT ANY WARRANTY; without even the implied warranty of
%    MERCHANTABILITY or FITNESS FOR A PARTICULAR PURPOSE.  See the
%    GNU General Public License for more details.
%
%    You should have received a copy of the GNU General Public License
%    along with this program; if not, write to the Free Software
%    Foundation, Inc., 59 Temple Place, Suite 330, Boston, MA  02111-1307  USA

% Thanks to the Xy-pic guys, Kristoffer H Rose, Ross Moore, and Daniel Müllner,
% for their help in making Qcircuit work with Xy-pic version 3.8.  
% Thanks also to Dave Clader, Andrew Childs, Rafael Possignolo, Tyson Williams,
% Sergio Boixo, Cris Moore, Jonas Anderson, and Stephan Mertens for helping us test 
% and/or develop the new version.

\usepackage{xy}
\xyoption{matrix}
\xyoption{frame}
\xyoption{arrow}
\xyoption{arc}

\usepackage{ifpdf}
\ifpdf
\else
\PackageWarningNoLine{Qcircuit}{Qcircuit is loading in Postscript mode.  The Xy-pic options ps and dvips will be loaded.  If you wish to use other Postscript drivers for Xy-pic, you must modify the code in Qcircuit.tex}
%    The following options load the drivers most commonly required to
%    get proper Postscript output from Xy-pic.  Should these fail to work,
%    try replacing the following two lines with some of the other options
%    given in the Xy-pic reference manual.
\xyoption{ps}
\xyoption{dvips}
\fi

% The following resets Xy-pic matrix alignment to the pre-3.8 default, as
% required by Qcircuit.
\entrymodifiers={!C\entrybox}

\newcommand{\bra}[1]{{\left\langle{#1}\right\vert}}
\newcommand{\ket}[1]{{\left\vert{#1}\right\rangle}}
    % Defines Dirac notation. %7/5/07 added extra braces so that the commands will work in subscripts.
\newcommand{\qw}[1][-1]{\ar @{-} [0,#1]}
    % Defines a wire that connects horizontally.  By default it connects to the object on the left of the current object.
    % WARNING: Wire commands must appear after the gate in any given entry.
\newcommand{\qwx}[1][-1]{\ar @{-} [#1,0]}
    % Defines a wire that connects vertically.  By default it connects to the object above the current object.
    % WARNING: Wire commands must appear after the gate in any given entry.
\newcommand{\cw}[1][-1]{\ar @{=} [0,#1]}
    % Defines a classical wire that connects horizontally.  By default it connects to the object on the left of the current object.
    % WARNING: Wire commands must appear after the gate in any given entry.
\newcommand{\cwx}[1][-1]{\ar @{=} [#1,0]}
    % Defines a classical wire that connects vertically.  By default it connects to the object above the current object.
    % WARNING: Wire commands must appear after the gate in any given entry.
\newcommand{\gate}[1]{*+<.6em>{#1} \POS ="i","i"+UR;"i"+UL **\dir{-};"i"+DL **\dir{-};"i"+DR **\dir{-};"i"+UR **\dir{-},"i" \qw}
    % Boxes the argument, making a gate.
\newcommand{\meter}{*=<1.8em,1.4em>{\xy ="j","j"-<.778em,.322em>;{"j"+<.778em,-.322em> \ellipse ur,_{}},"j"-<0em,.4em>;p+<.5em,.9em> **\dir{-},"j"+<2.2em,2.2em>*{},"j"-<2.2em,2.2em>*{} \endxy} \POS ="i","i"+UR;"i"+UL **\dir{-};"i"+DL **\dir{-};"i"+DR **\dir{-};"i"+UR **\dir{-},"i" \qw}
    % Inserts a measurement meter.
    % In case you're wondering, the constants .778em and .322em specify
    % one quarter of a circle with radius 1.1em.
    % The points added at + and - <2.2em,2.2em> are there to strech the
    % canvas, ensuring that the size is unaffected by erratic spacing issues
    % with the arc.
\newcommand{\measure}[1]{*+[F-:<.9em>]{#1} \qw}
    % Inserts a measurement bubble with user defined text.
\newcommand{\measuretab}[1]{*{\xy*+<.6em>{#1}="e";"e"+UL;"e"+UR **\dir{-};"e"+DR **\dir{-};"e"+DL **\dir{-};"e"+LC-<.5em,0em> **\dir{-};"e"+UL **\dir{-} \endxy} \qw}
    % Inserts a measurement tab with user defined text.
\newcommand{\measureD}[1]{*{\xy*+=<0em,.1em>{#1}="e";"e"+UR+<0em,.25em>;"e"+UL+<-.5em,.25em> **\dir{-};"e"+DL+<-.5em,-.25em> **\dir{-};"e"+DR+<0em,-.25em> **\dir{-};{"e"+UR+<0em,.25em>\ellipse^{}};"e"+C:,+(0,1)*{} \endxy} \qw}
    % Inserts a D-shaped measurement gate with user defined text.
\newcommand{\multimeasure}[2]{*+<1em,.9em>{\hphantom{#2}} \qw \POS[0,0].[#1,0];p !C *{#2},p \drop\frm<.9em>{-}}
    % Draws a multiple qubit measurement bubble starting at the current position and spanning #1 additional gates below.
    % #2 gives the label for the gate.
    % You must use an argument of the same width as #2 in \ghost for the wires to connect properly on the lower lines.
\newcommand{\multimeasureD}[2]{*+<1em,.9em>{\hphantom{#2}} \POS [0,0]="i",[0,0].[#1,0]="e",!C *{#2},"e"+UR-<.8em,0em>;"e"+UL **\dir{-};"e"+DL **\dir{-};"e"+DR+<-.8em,0em> **\dir{-};{"e"+DR+<0em,.8em>\ellipse^{}};"e"+UR+<0em,-.8em> **\dir{-};{"e"+UR-<.8em,0em>\ellipse^{}},"i" \qw}
    % Draws a multiple qubit D-shaped measurement gate starting at the current position and spanning #1 additional gates below.
    % #2 gives the label for the gate.
    % You must use an argument of the same width as #2 in \ghost for the wires to connect properly on the lower lines.
\newcommand{\control}{*!<0em,.025em>-=-<.2em>{\bullet}}
    % Inserts an unconnected control.
\newcommand{\controlo}{*+<.01em>{\xy -<.095em>*\xycircle<.19em>{} \endxy}}
    % Inserts a unconnected control-on-0.
\newcommand{\ctrl}[1]{\control \qwx[#1] \qw}
    % Inserts a control and connects it to the object #1 wires below.
\newcommand{\ctrlo}[1]{\controlo \qwx[#1] \qw}
    % Inserts a control-on-0 and connects it to the object #1 wires below.
\newcommand{\targ}{*+<.02em,.02em>{\xy ="i","i"-<.39em,0em>;"i"+<.39em,0em> **\dir{-}, "i"-<0em,.39em>;"i"+<0em,.39em> **\dir{-},"i"*\xycircle<.4em>{} \endxy} \qw}
    % Inserts a CNOT target.
\newcommand{\qswap}{*=<0em>{\times} \qw}
    % Inserts half a swap gate.
    % Must be connected to the other swap with \qwx.
\newcommand{\multigate}[2]{*+<1em,.9em>{\hphantom{#2}} \POS [0,0]="i",[0,0].[#1,0]="e",!C *{#2},"e"+UR;"e"+UL **\dir{-};"e"+DL **\dir{-};"e"+DR **\dir{-};"e"+UR **\dir{-},"i" \qw}
    % Draws a multiple qubit gate starting at the current position and spanning #1 additional gates below.
    % #2 gives the label for the gate.
    % You must use an argument of the same width as #2 in \ghost for the wires to connect properly on the lower lines.
\newcommand{\ghost}[1]{*+<1em,.9em>{\hphantom{#1}} \qw}
    % Leaves space for \multigate on wires other than the one on which \multigate appears.  Without this command wires will cross your gate.
    % #1 should match the second argument in the corresponding \multigate.
\newcommand{\push}[1]{*{#1}}
    % Inserts #1, overriding the default that causes entries to have zero size.  This command takes the place of a gate.
    % Like a gate, it must precede any wire commands.
    % \push is useful for forcing columns apart.
    % NOTE: It might be useful to know that a gate is about 1.3 times the height of its contents.  I.e. \gate{M} is 1.3em tall.
    % WARNING: \push must appear before any wire commands and may not appear in an entry with a gate or label.
\newcommand{\gategroup}[6]{\POS"#1,#2"."#3,#2"."#1,#4"."#3,#4"!C*+<#5>\frm{#6}}
    % Constructs a box or bracket enclosing the square block spanning rows #1-#3 and columns=#2-#4.
    % The block is given a margin #5/2, so #5 should be a valid length.
    % #6 can take the following arguments -- or . or _\} or ^\} or \{ or \} or _) or ^) or ( or ) where the first two options yield dashed and
    % dotted boxes respectively, and the last eight options yield bottom, top, left, and right braces of the curly or normal variety.  See the Xy-pic reference manual for more options.
    % \gategroup can appear at the end of any gate entry, but it's good form to pick either the last entry or one of the corner gates.
    % BUG: \gategroup uses the four corner gates to determine the size of the bounding box.  Other gates may stick out of that box.  See \prop.

\newcommand{\rstick}[1]{*!L!<-.5em,0em>=<0em>{#1}}
    % Centers the left side of #1 in the cell.  Intended for lining up wire labels.  Note that non-gates have default size zero.
\newcommand{\lstick}[1]{*!R!<.5em,0em>=<0em>{#1}}
    % Centers the right side of #1 in the cell.  Intended for lining up wire labels.  Note that non-gates have default size zero.
\newcommand{\ustick}[1]{*!D!<0em,-.5em>=<0em>{#1}}
    % Centers the bottom of #1 in the cell.  Intended for lining up wire labels.  Note that non-gates have default size zero.
\newcommand{\dstick}[1]{*!U!<0em,.5em>=<0em>{#1}}
    % Centers the top of #1 in the cell.  Intended for lining up wire labels.  Note that non-gates have default size zero.
\newcommand{\Qcircuit}{\xymatrix @*=<0em>}
    % Defines \Qcircuit as an \xymatrix with entries of default size 0em.
\newcommand{\link}[2]{\ar @{-} [#1,#2]}
    % Draws a wire or connecting line to the element #1 rows down and #2 columns forward.
\newcommand{\pureghost}[1]{*+<1em,.9em>{\hphantom{#1}}}
    % Same as \ghost except it omits the wire leading to the left. 

\usepackage{color}
\usepackage{amsmath,amsthm,amsbsy,amssymb,dsfont}
\usepackage{mathtools}
\usepackage[colorlinks=true,urlcolor=webblue,linkcolor=webgreen,filecolor=webblue,citecolor=webgreen,pdfpagemode=UseOutlines,pdfstartview=FitH,pdfpagelayout=OneColumn,bookmarks=true]{hyperref}
\usepackage{fullpage}
\usepackage{doi}
\usepackage[numbers]{natbib}
\usepackage{microtype}
\usepackage{xspace}
\usepackage{tikz}
\usepackage{bigints}
\usepackage{mathptmx}
\usepackage{palatino}
\usepackage{eucal}


\hypersetup{pdfauthor={Gorjan Alagic}}

\definecolor{webgreen}{rgb}{0,.5,0}
\definecolor{webblue}{rgb}{0,0,.5}

\DeclareMathOperator{\tr}{Tr}
\DeclareMathOperator*{\Exp}{\mathbb{E}}

\numberwithin{equation}{section}

\newtheorem{theorem}{Theorem}
\newtheorem{prop}{Proposition}
\newtheorem{proposition}{Proposition}
\newtheorem{lemma}[theorem]{Lemma}
\newtheorem{conjecture}{Conjecture}
\newtheorem{claim}[theorem]{Claim}
\newtheorem{corollary}[theorem]{Corollary}
\newtheorem{definition}{Definition}
\newtheorem{question}{Question}

%% Define a satisfactory mathbb 1.
\newcommand{\one}{\mathds 1}
\renewcommand{\vec}[1]{\mathbf{#1}}
\newcommand{\D}{\mathsf{D}}
\DeclareMathOperator{\Ind}{Ind}
\DeclareMathOperator{\End}{End}
\newcommand{\op}{\operatorname{op}}
\newcommand{\opn}{\operatorname}
\newcommand{\inter}{\mathfrak{I}}
\newcommand{\eqdef}{\triangleq}
\newcommand{\E}{\mathbb{E}}
\newcommand{\C}{\mathbb{C}}
\newcommand{\R}{\mathbb{R}}
\newcommand{\Z}{\mathbb{Z}}
\newcommand{\F}{\mathbb{F}}
\newcommand{\N}{\mathbb{N}}
\newcommand{\U}{\textsf{U}}
\newcommand{\outerprod}[2]{|#1\rangle\langle #2|}
\newcommand{\innerprod}[2]{\langle #1, #2\rangle}
%\newcommand{\ket}[1]{|#1\rangle}
%\newcommand{\bra}[1]{\langle #1|}
\newcommand{\Hom}{\operatorname{Hom}}
\newcommand{\GL}{\textsf{GL}}
\newcommand{\SU}{\textsf{SU}}


\newcommand{\expref}[2]{\texorpdfstring{\hyperref[#2]{#1~\ref{#2}}}{#1~\ref{#2}}}
\newcommand{\expreft}[2]{\texorpdfstring{\hyperref[#2]{#1}}{#1}}

\newcommand{\revise}[1]{}
\renewcommand{\vec}[1]{\mathbf{#1}}

\newcommand{\band}[1]{\mathcal{#1}}
\newcommand{\error}{\varepsilon}
\newcommand{\recon}{R}
\newcommand{\idim}{\dim_{\opn{I}}}

%%% added by GA  Sept 2015
\newcommand{\algo}{\mathcal}
\newcommand{\negl}{\opn{negl}}
\newcommand{\KeyGen}{\ensuremath{\mathsf{KeyGen}}\xspace}
\newcommand{\Enc}{\ensuremath{\mathsf{Enc}}\xspace}
\newcommand{\Dec}{\ensuremath{\mathsf{Dec}}\xspace}
\newcommand{\Homorcl}{\ensuremath{\mathsf{Hom}}\xspace}
\newcommand{\Mint}{\ensuremath{\mathsf{Mint}}\xspace}
\newcommand{\Verify}{\ensuremath{\mathsf{Verify}}\xspace}
\newcommand{\inrand}{\in_R} 
\newcommand{\prob}{\opn{Pr}}
\newcommand{\states}{\mathfrak D}
\newcommand\supp{\textbf{supp}}
\newcommand\Eval{\ensuremath{\mathsf{Eval}}\xspace}
%%%

\newcommand{\ga}[1]{{ \textcolor{purple}{(Gorjan:  #1)}}{}}
\newcommand{\wf}[1]{{ \textcolor{orange}{(Bill:  #1)}}{}}


\begin{document}


\title{Quantum obfuscation}
\author{Gorjan Alagic and Bill Fefferman}
\maketitle
\begin{abstract}
Encryption of data is fundamental to secure communication. Beyond encryption of data lies \emph{obfuscation}, i.e., encryption of functionality. It has been known for some time that the most powerful classical obfuscation, so-called ``black-box obfuscation,'' is impossible. In this work, we initialize the rigorous study of obfuscating programs \emph{via quantum-mechanical means.} We prove quantum analogues of several foundational results in obfuscation, including the aforementioned black-box impossibility result.

In its most powerful ``quantum black-box'' instantiation, a quantum obfuscator would turn a description of a quantum program $f$ into a quantum state $\rho_f$, such that anyone in possession of $\rho_f$ can repeatedly evaluate $f$ on inputs of their choice, but never learn \emph{anything else} about the original program. We formalize this notion of obfuscation, and prove an impossibility result: such obfuscation is only possible in a setting where the adversary never has access to more than one obfuscation (of either the same program, or of different programs.) Our proof involves a novel technical idea: chosen-ciphertext-secure encryption for quantum states. In addition, we show that some applications of obfuscation still appear possible in spite of our impossibility result. These include CPA-secure encryption, fully-homomorphic encryption, and quantum money. 

We also define quantum versions of indistinguishability obfuscation and best-possible obfuscation. We then show that these notions are equivalent, and that their perfect and statistical variants are impossible to achieve. The remaining (i.e., computational) variant would still have an application of interest: witness encryption for QMA$_1$.
\end{abstract}


%%%%%%%%%%%%%%%%%%%%%%
\section{Introduction}\label{sec:intro}
%%%%%%%%%%%%%%%%%%%%%%

\subsection{Obfuscation.} Obfuscation is \emph{encryption of functionality}, and is arguably the most powerful cryptographic ability that may yet be possible. It implies (with some caveats) almost any other cryptographic construction imaginable. It could be used to protect intellectual property in software, and provide secure software patching. In all of these applications, an \emph{obfuscator} is an efficient algorithm which rewrites programs so that they satisfy:
\begin{enumerate}\label{def:obf-informal}
\item \emph{functional equivalence:} input/output functionality is unchanged;
\item \emph{polynomial slowdown:} efficiency is maintained;
\item \emph{obfuscation:} the code of the output program is ``hard to understand.''
\end{enumerate}
The last condition admits several rigorous formulations. The strongest is the so-called ``virtual black-box'' condition, which says that the obfuscated program is no more useful than a formless, impenetrable box which simply accepts inputs and produces outputs. 

\subsection{Classical status.} The first major result in classical obfuscation was the 2001 proof by Barak et al. that virtual black-box obfuscation is impossible, and that many of the applications are impossible to achieve generically~\cite{BGIRSVY01, BGIRSVY12}. An important step in formulating alternative notions of obfuscation was taken by Goldwasser and Rothblum; they defined \emph{indistinguishability obfuscation} and \emph{best-possible obfuscation}~\cite{GR07}. Indistinguishability requires that functionally-equivalent circuits are mapped to indistinguishable distributions (over circuits); best-possible requires that circuits are mapped to the ``least leaky'' functionally equivalent circuit. Both definitions have perfect, statistical, and computational variants. In~\cite{GR07} it was shown that indistinguishability and best-possible are equivalent, and that the perfect and statistical versions are impossible (barring PH collapse)~\cite{GR07}. 

In 2013, in a breakthrough result, Garg et al. proposed a convincing candidate for the one remaining possibility: computational indistinguishability obfuscation. Their proposal was based on the hardness of a certain problem in multilinear maps~\cite{GGHRSW13}. It was accompanied by another breakthrough, which showed a range of applications so wide, that indistinguishability obfuscation was proposed as a new ``\,'central hub' for cryptography''~\cite{SW14}. These two breakthroughs were followed by a flurry of new activity in the area, including several new proposals and applications~\cite{BGKPS14, BCCGKPR14, BZ14, BR14, GGHW14, HSW14}. Unfortunately, the quantum security of the underlying hardness assumptions has recently been put into doubt~\cite{Pei15}.

\subsection{Quantum status.} Quantum obfuscation is essentially an unexplored topic, and the present work appears to be the first rigorous treatment of the foundational questions. The question of whether quantum obfuscation is possible was posed as one of Scott Aaronson's ``semi-grand challenges'' for quantum computation~\cite{Aar05}. Since so little work on quantum obfuscation has appeared, we briefly discuss some related results. In~\cite{Aar09}, Aaronson proposed a \emph{complexity-theoretic no-cloning theorem}; this theorem was eventually proved in a paper on quantum money~\cite{AC12}. In related work, Mosca and Stebila showed how to use Aaronson's theorem to give a simple black-box quantum money scheme, and suggested the possibility of using a quantum circuit obfuscator in place of the black box~\cite{MS10}. More recently, Alagic, Jeffery and Jordan proposed obfuscators for both classical (reversible) circuits and quantum circuits, based on ideas from topological quantum computation~\cite{ASS14}. The proposed obfuscator satisfies indistinguishability for a restricted set of circuit equivalences; its usefulness is unclear at this time.

%%%%%%%%%%%%%%%%%%%%%%
\section{Our results}\label{sec:intro}
%%%%%%%%%%%%%%%%%%%%%%

We now summarize our contributions; details are explained in the full technical version. In what follows, poly-time algorithms will be called PT, PPT, or QPT; these mean (respectively) classical deterministic, probabilistic, and quantum. The unitary operator implemented by a quantum circuit $C$ is denoted $U_C$. Functions decaying faster than any inverse-polynomial are denoted $\negl(\cdot)$.

\subsection{Quantum black-box obfuscation.} 
%%%%

\begin{definition}\label{def:vbb-obfuscator}
A \textbf{black-box quantum obfuscator} is a quantum algorithm $\algo O$ and a QPT $\algo J$ such that for any $n$-qubit quantum circuit $C$, $\mathcal O(C)$ is a poly$(n)$-qubit quantum state satisfying:
\begin{enumerate}
\item (functional equivalence) $\bigl\| \algo J ( \algo O(C) \otimes \rho ) - U_C \rho U_C^\dagger \bigr\|_\emph{tr} \leq \negl(n)$ for all $n$-qubit states $\rho$;
\item (virtual black-box) $\forall$ QPT $\mathcal A$ $\exists$ QPT $\mathcal S^{U_C}$ such that
$| \emph{Pr}[\mathcal A(\mathcal O(C)) = 1] - \emph{Pr}[\mathcal S^{U_C}(0^n) = 1]| \leq \negl(n).$
\end{enumerate}
\end{definition}

\noindent We emphasize that $\algo O$ need not be poly-time, although its outputs must be poly-sized. A departure from the classical definition is the addition of the ``interpreter'' $\algo J$. It is natural since there must be \emph{some efficient way} to run $U_C$ using the state $\algo O(C)$; whatever it is, we denote it by $\algo J$. We prove the following impossibility results for these obfuscators.

\begin{theorem}\label{thm:main}
Black-box quantum obfuscation is impossible for pairs: an adversary with access to two outputs\footnote{If the two outputs came from the same input circuit, then impossibility only holds if the output states are \textbf{reusable}, in the sense that $\algo J$ outputs another state which again satisfies functional equivalence. This is weaker than cloneability.} of the obfuscator can violate the black-box condition.
\end{theorem}

\begin{corollary}\label{cor:main}
Black-box obfuscation of quantum circuits into quantum circuits is impossible.
\end{corollary}

\noindent To prove the theorem, we give an explicit construction of a circuit family $\mathcal C$, such that no family of states is an obfuscation of $\mathcal C$. \expref{Corollary}{cor:main} is a quantum generalization of the main result of Barak et al.~\cite{BGIRSVY01}. We emphasize that, perhaps surprisingly, one cannot conclude \expref{Corollary}{cor:main} from \cite{BGIRSVY01} by appealing to the fact that classical functions are a special case of quantum functions; for one, the adversaries and simulators are now also quantum. Generalizing to the full \expref{Theorem}{thm:main} is also nontrivial. The main overall technical obstacle is to show that the states output by the obfuscator can be ``executed on one another,'' without also revealing certain secrets to simulators having only black-box access to the original circuits. A crucial ingredient in overcoming this obstacle is a notion of \emph{chosen-ciphertext-secure quantum encryption.} To achieve this, we show:

\begin{theorem}\label{thm:cca}
If quantum-secure one-way functions (qOWF) exist, then so do IND-CCA1-secure symmetric-key quantum encryption schemes (qSKE).
\end{theorem}

\noindent The above requires several new definitions (e.g., IND-CCA1 for quantum encryption) which we also provide. A complete treatment of the subject of quantum encryption with computational assumptions will soon appear in a joint work~\cite{ABFGSS15}.

In addition, we provide several applications of quantum black-box obfuscation, which still appear feasible (in some form) in spite of \expref{Theorem}{thm:main}. They are briefly outlined as follows.
\begin{enumerate}
\item \textbf{IND-CPA-secure encryption.} Crucially, this demands only a quantum black-box obfuscator, but \emph{not} one-way functions (quantum-secure or otherwise.)
\item \textbf{qOWF imply IND-CPA public-key homomorphic encryption.} Start with qSKE via \expref{Theorem}{thm:cca}; public keys are obfuscations of encrypt circuits; evaluation keys are obfuscations of a universal decrypt-compute-encrypt circuit. Trapdoor permutations are not required.
\item \textbf{Public-key quantum money}. This was proposed by Mosca and Stebila~\cite{MS10}, using a result of Aaronson and Christiano~\cite{Aar09, AC12}. As we show, a certain adaptation survives \expref{Theorem}{thm:main}.
\end{enumerate}
\noindent We emphasize that applications 1 and 2 also work for achieving \emph{classical functionality} from a quantum obfuscator; however, they use quantum ciphertexts and quantum keys---another approach not considered before.

\subsection{Quantum indistinguishability obfuscation.}
%%%%

Following the classical approach of Goldwasser and Rothblum~\cite{GR07}, we define a version of \expref{Definition}{def:vbb-obfuscator} which guarantees indistinguishability of obfuscator outputs, by replacing the black-box condition (2) with:
\begin{enumerate}
\setcounter{enumi}{1}
\item \emph{(indistinguishability) if $\|U_{C_1} - U_{C_2}\| \leq \negl(n)$, then $\|\mathcal O(C_1) - \mathcal O(C_2)\|_{*} \leq \negl(n)$.}
\end{enumerate}
We also define a notion of \emph{quantum best-possible obfuscation}:
\begin{enumerate}
\setcounter{enumi}{1}
\item \emph{(best-possible) if $\|U_{C_1} - U_{C_2}\| \leq \negl(n)$, then for all QPT $\mathcal A$ there exists a QPT $\mathcal S$ satisfying}
$\| \mathcal A(\mathcal O(C_1)) - \mathcal S(C_2) \|_{*} \leq \negl(n)$.
\end{enumerate}
\noindent Both definitions above have three variants, depending on the nature of the norm $\| \cdot \|_{*}$: perfect, statistical, and computational (against QPTs). We prove the following equivalence result.

\begin{theorem}
A QPT is an indistinguishability obfuscator if and only if it is a best-possible obfuscator.
\end{theorem}

\noindent We also show that these definitions are only achievable when the distinguishability is guaranteed only against computationally bounded (quantum) adversaries.

\begin{theorem}
If quantum statistical-indistinguishability obfuscators exist, then coQMA $\subseteq$ QSZK.
\end{theorem}

This means that, just as in the classical world, computational indistinguishability is the only surviving variant. We end with an application for such an obfuscator: quantum witness encryption for QMA$_{1}$. Witness encryption for a QMA$_{1}$ language $L$ provides for encryption of a plaintext $x$ to a potential instance $l$. The security guarantee is that (i.) if $l \in L$, then any valid witness allows for decryption of $x$, and (ii.) if $l \notin L$, then encryptions of different plaintexts are indistinguishable. 

\begin{theorem}
Quantum computational-indistinguishability obfuscation implies witness encryption for QMA$_1$.
\end{theorem}

\noindent Classical witness encryption is known to have numerous applications~\cite{GGSW13}. We conjecture that many of the other recently discovered classical applications of computational indistinguishability obfuscation (see, e.g., ~\cite{SW14}) also have interesting quantum analogues or extensions.

%%%%%%%%%%%%%%%%%%%%%%%%%%%%%%%%
%Bill: Changed \bibstyle location (wouldn't compile for me otherwise)
\bibliographystyle{plainnat}
\bibliography{QuantumCrypto}
%%%%%%%%%%%%%%%%%%%%%%%%%%%%%%%%

\end{document}


