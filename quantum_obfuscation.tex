\documentclass[11pt]{article}

%    Q-circuit version 2
%    Copyright (C) 2004  Steve Flammia & Bryan Eastin
%    Last modified on: 9/16/2011
%
%    This program is free software; you can redistribute it and/or modify
%    it under the terms of the GNU General Public License as published by
%    the Free Software Foundation; either version 2 of the License, or
%    (at your option) any later version.
%
%    This program is distributed in the hope that it will be useful,
%    but WITHOUT ANY WARRANTY; without even the implied warranty of
%    MERCHANTABILITY or FITNESS FOR A PARTICULAR PURPOSE.  See the
%    GNU General Public License for more details.
%
%    You should have received a copy of the GNU General Public License
%    along with this program; if not, write to the Free Software
%    Foundation, Inc., 59 Temple Place, Suite 330, Boston, MA  02111-1307  USA

% Thanks to the Xy-pic guys, Kristoffer H Rose, Ross Moore, and Daniel Müllner,
% for their help in making Qcircuit work with Xy-pic version 3.8.  
% Thanks also to Dave Clader, Andrew Childs, Rafael Possignolo, Tyson Williams,
% Sergio Boixo, Cris Moore, Jonas Anderson, and Stephan Mertens for helping us test 
% and/or develop the new version.

\usepackage{xy}
\xyoption{matrix}
\xyoption{frame}
\xyoption{arrow}
\xyoption{arc}

\usepackage{ifpdf}
\ifpdf
\else
\PackageWarningNoLine{Qcircuit}{Qcircuit is loading in Postscript mode.  The Xy-pic options ps and dvips will be loaded.  If you wish to use other Postscript drivers for Xy-pic, you must modify the code in Qcircuit.tex}
%    The following options load the drivers most commonly required to
%    get proper Postscript output from Xy-pic.  Should these fail to work,
%    try replacing the following two lines with some of the other options
%    given in the Xy-pic reference manual.
\xyoption{ps}
\xyoption{dvips}
\fi

% The following resets Xy-pic matrix alignment to the pre-3.8 default, as
% required by Qcircuit.
\entrymodifiers={!C\entrybox}

\newcommand{\bra}[1]{{\left\langle{#1}\right\vert}}
\newcommand{\ket}[1]{{\left\vert{#1}\right\rangle}}
    % Defines Dirac notation. %7/5/07 added extra braces so that the commands will work in subscripts.
\newcommand{\qw}[1][-1]{\ar @{-} [0,#1]}
    % Defines a wire that connects horizontally.  By default it connects to the object on the left of the current object.
    % WARNING: Wire commands must appear after the gate in any given entry.
\newcommand{\qwx}[1][-1]{\ar @{-} [#1,0]}
    % Defines a wire that connects vertically.  By default it connects to the object above the current object.
    % WARNING: Wire commands must appear after the gate in any given entry.
\newcommand{\cw}[1][-1]{\ar @{=} [0,#1]}
    % Defines a classical wire that connects horizontally.  By default it connects to the object on the left of the current object.
    % WARNING: Wire commands must appear after the gate in any given entry.
\newcommand{\cwx}[1][-1]{\ar @{=} [#1,0]}
    % Defines a classical wire that connects vertically.  By default it connects to the object above the current object.
    % WARNING: Wire commands must appear after the gate in any given entry.
\newcommand{\gate}[1]{*+<.6em>{#1} \POS ="i","i"+UR;"i"+UL **\dir{-};"i"+DL **\dir{-};"i"+DR **\dir{-};"i"+UR **\dir{-},"i" \qw}
    % Boxes the argument, making a gate.
\newcommand{\meter}{*=<1.8em,1.4em>{\xy ="j","j"-<.778em,.322em>;{"j"+<.778em,-.322em> \ellipse ur,_{}},"j"-<0em,.4em>;p+<.5em,.9em> **\dir{-},"j"+<2.2em,2.2em>*{},"j"-<2.2em,2.2em>*{} \endxy} \POS ="i","i"+UR;"i"+UL **\dir{-};"i"+DL **\dir{-};"i"+DR **\dir{-};"i"+UR **\dir{-},"i" \qw}
    % Inserts a measurement meter.
    % In case you're wondering, the constants .778em and .322em specify
    % one quarter of a circle with radius 1.1em.
    % The points added at + and - <2.2em,2.2em> are there to strech the
    % canvas, ensuring that the size is unaffected by erratic spacing issues
    % with the arc.
\newcommand{\measure}[1]{*+[F-:<.9em>]{#1} \qw}
    % Inserts a measurement bubble with user defined text.
\newcommand{\measuretab}[1]{*{\xy*+<.6em>{#1}="e";"e"+UL;"e"+UR **\dir{-};"e"+DR **\dir{-};"e"+DL **\dir{-};"e"+LC-<.5em,0em> **\dir{-};"e"+UL **\dir{-} \endxy} \qw}
    % Inserts a measurement tab with user defined text.
\newcommand{\measureD}[1]{*{\xy*+=<0em,.1em>{#1}="e";"e"+UR+<0em,.25em>;"e"+UL+<-.5em,.25em> **\dir{-};"e"+DL+<-.5em,-.25em> **\dir{-};"e"+DR+<0em,-.25em> **\dir{-};{"e"+UR+<0em,.25em>\ellipse^{}};"e"+C:,+(0,1)*{} \endxy} \qw}
    % Inserts a D-shaped measurement gate with user defined text.
\newcommand{\multimeasure}[2]{*+<1em,.9em>{\hphantom{#2}} \qw \POS[0,0].[#1,0];p !C *{#2},p \drop\frm<.9em>{-}}
    % Draws a multiple qubit measurement bubble starting at the current position and spanning #1 additional gates below.
    % #2 gives the label for the gate.
    % You must use an argument of the same width as #2 in \ghost for the wires to connect properly on the lower lines.
\newcommand{\multimeasureD}[2]{*+<1em,.9em>{\hphantom{#2}} \POS [0,0]="i",[0,0].[#1,0]="e",!C *{#2},"e"+UR-<.8em,0em>;"e"+UL **\dir{-};"e"+DL **\dir{-};"e"+DR+<-.8em,0em> **\dir{-};{"e"+DR+<0em,.8em>\ellipse^{}};"e"+UR+<0em,-.8em> **\dir{-};{"e"+UR-<.8em,0em>\ellipse^{}},"i" \qw}
    % Draws a multiple qubit D-shaped measurement gate starting at the current position and spanning #1 additional gates below.
    % #2 gives the label for the gate.
    % You must use an argument of the same width as #2 in \ghost for the wires to connect properly on the lower lines.
\newcommand{\control}{*!<0em,.025em>-=-<.2em>{\bullet}}
    % Inserts an unconnected control.
\newcommand{\controlo}{*+<.01em>{\xy -<.095em>*\xycircle<.19em>{} \endxy}}
    % Inserts a unconnected control-on-0.
\newcommand{\ctrl}[1]{\control \qwx[#1] \qw}
    % Inserts a control and connects it to the object #1 wires below.
\newcommand{\ctrlo}[1]{\controlo \qwx[#1] \qw}
    % Inserts a control-on-0 and connects it to the object #1 wires below.
\newcommand{\targ}{*+<.02em,.02em>{\xy ="i","i"-<.39em,0em>;"i"+<.39em,0em> **\dir{-}, "i"-<0em,.39em>;"i"+<0em,.39em> **\dir{-},"i"*\xycircle<.4em>{} \endxy} \qw}
    % Inserts a CNOT target.
\newcommand{\qswap}{*=<0em>{\times} \qw}
    % Inserts half a swap gate.
    % Must be connected to the other swap with \qwx.
\newcommand{\multigate}[2]{*+<1em,.9em>{\hphantom{#2}} \POS [0,0]="i",[0,0].[#1,0]="e",!C *{#2},"e"+UR;"e"+UL **\dir{-};"e"+DL **\dir{-};"e"+DR **\dir{-};"e"+UR **\dir{-},"i" \qw}
    % Draws a multiple qubit gate starting at the current position and spanning #1 additional gates below.
    % #2 gives the label for the gate.
    % You must use an argument of the same width as #2 in \ghost for the wires to connect properly on the lower lines.
\newcommand{\ghost}[1]{*+<1em,.9em>{\hphantom{#1}} \qw}
    % Leaves space for \multigate on wires other than the one on which \multigate appears.  Without this command wires will cross your gate.
    % #1 should match the second argument in the corresponding \multigate.
\newcommand{\push}[1]{*{#1}}
    % Inserts #1, overriding the default that causes entries to have zero size.  This command takes the place of a gate.
    % Like a gate, it must precede any wire commands.
    % \push is useful for forcing columns apart.
    % NOTE: It might be useful to know that a gate is about 1.3 times the height of its contents.  I.e. \gate{M} is 1.3em tall.
    % WARNING: \push must appear before any wire commands and may not appear in an entry with a gate or label.
\newcommand{\gategroup}[6]{\POS"#1,#2"."#3,#2"."#1,#4"."#3,#4"!C*+<#5>\frm{#6}}
    % Constructs a box or bracket enclosing the square block spanning rows #1-#3 and columns=#2-#4.
    % The block is given a margin #5/2, so #5 should be a valid length.
    % #6 can take the following arguments -- or . or _\} or ^\} or \{ or \} or _) or ^) or ( or ) where the first two options yield dashed and
    % dotted boxes respectively, and the last eight options yield bottom, top, left, and right braces of the curly or normal variety.  See the Xy-pic reference manual for more options.
    % \gategroup can appear at the end of any gate entry, but it's good form to pick either the last entry or one of the corner gates.
    % BUG: \gategroup uses the four corner gates to determine the size of the bounding box.  Other gates may stick out of that box.  See \prop.

\newcommand{\rstick}[1]{*!L!<-.5em,0em>=<0em>{#1}}
    % Centers the left side of #1 in the cell.  Intended for lining up wire labels.  Note that non-gates have default size zero.
\newcommand{\lstick}[1]{*!R!<.5em,0em>=<0em>{#1}}
    % Centers the right side of #1 in the cell.  Intended for lining up wire labels.  Note that non-gates have default size zero.
\newcommand{\ustick}[1]{*!D!<0em,-.5em>=<0em>{#1}}
    % Centers the bottom of #1 in the cell.  Intended for lining up wire labels.  Note that non-gates have default size zero.
\newcommand{\dstick}[1]{*!U!<0em,.5em>=<0em>{#1}}
    % Centers the top of #1 in the cell.  Intended for lining up wire labels.  Note that non-gates have default size zero.
\newcommand{\Qcircuit}{\xymatrix @*=<0em>}
    % Defines \Qcircuit as an \xymatrix with entries of default size 0em.
\newcommand{\link}[2]{\ar @{-} [#1,#2]}
    % Draws a wire or connecting line to the element #1 rows down and #2 columns forward.
\newcommand{\pureghost}[1]{*+<1em,.9em>{\hphantom{#1}}}
    % Same as \ghost except it omits the wire leading to the left. 

\usepackage{color}
\usepackage{amsmath,amsthm,amsbsy,amssymb,dsfont}
\usepackage{mathtools}
\usepackage[colorlinks=true,urlcolor=webblue,linkcolor=webgreen,filecolor=webblue,citecolor=webgreen,pdfpagemode=UseOutlines,pdfstartview=FitH,pdfpagelayout=OneColumn,bookmarks=true]{hyperref}
\usepackage{fullpage}
\usepackage{doi}
\usepackage[numbers]{natbib}
\usepackage{microtype}
\usepackage{xspace}
\usepackage{tikz}
\usepackage{bigints}
\usepackage{mathptmx}
\usepackage{palatino}
\usepackage{eucal}

\hypersetup{pdfauthor={Gorjan Alagic}}

\definecolor{webgreen}{rgb}{0,.5,0}
\definecolor{webblue}{rgb}{0,0,.5}

\DeclareMathOperator{\tr}{Tr}
\DeclareMathOperator*{\Exp}{\mathbb{E}}

\numberwithin{equation}{section}

\newtheorem{theorem}{Theorem}
\newtheorem{prop}{Proposition}
\newtheorem{proposition}{Proposition}
\newtheorem{lemma}[theorem]{Lemma}
\newtheorem{conjecture}{Conjecture}
\newtheorem{claim}[theorem]{Claim}
\newtheorem{corollary}[theorem]{Corollary}
\newtheorem{definition}{Definition}
\newtheorem{question}{Question}

%% Define a satisfactory mathbb 1.
\newcommand{\one}{\mathds 1}
\renewcommand{\vec}[1]{\mathbf{#1}}
\newcommand{\D}{\mathsf{D}}
\DeclareMathOperator{\Ind}{Ind}
\DeclareMathOperator{\End}{End}
\newcommand{\op}{\operatorname{op}}
\newcommand{\opn}{\operatorname}
\newcommand{\inter}{\mathfrak{I}}
\newcommand{\eqdef}{\triangleq}
\newcommand{\E}{\mathbb{E}}
\newcommand{\C}{\mathbb{C}}
\newcommand{\R}{\mathbb{R}}
\newcommand{\Z}{\mathbb{Z}}
\newcommand{\F}{\mathbb{F}}
\newcommand{\N}{\mathbb{N}}
\newcommand{\U}{\textsf{U}}
\newcommand{\outerprod}[2]{|#1\rangle\langle #2|}
\newcommand{\innerprod}[2]{\langle #1, #2\rangle}
%\newcommand{\ket}[1]{|#1\rangle}
%\newcommand{\bra}[1]{\langle #1|}
\newcommand{\Hom}{\operatorname{Hom}}
\newcommand{\GL}{\textsf{GL}}
\newcommand{\SU}{\textsf{SU}}


\newcommand{\expref}[2]{\texorpdfstring{\hyperref[#2]{#1~\ref{#2}}}{#1~\ref{#2}}}
\newcommand{\expreft}[2]{\texorpdfstring{\hyperref[#2]{#1}}{#1}}

\newcommand{\revise}[1]{}
\renewcommand{\vec}[1]{\mathbf{#1}}

\newcommand{\band}[1]{\mathcal{#1}}
\newcommand{\error}{\varepsilon}
\newcommand{\recon}{R}
\newcommand{\idim}{\dim_{\opn{I}}}

%%% added by GA  Sept 2015
\newcommand{\algo}{\mathcal}
\newcommand{\negl}{\opn{negl}}
\newcommand{\KeyGen}{\ensuremath{\mathsf{KeyGen}}\xspace}
\newcommand{\Enc}{\ensuremath{\mathsf{Enc}}\xspace}
\newcommand{\Dec}{\ensuremath{\mathsf{Dec}}\xspace}
\newcommand{\Homorcl}{\ensuremath{\mathsf{Hom}}\xspace}
\newcommand{\Mint}{\ensuremath{\mathsf{Mint}}\xspace}
\newcommand{\Verify}{\ensuremath{\mathsf{Verify}}\xspace}
\newcommand{\inrand}{\in_R} 
\newcommand{\prob}{\opn{Pr}}
\newcommand{\states}{\mathfrak D}
\newcommand\supp{\textbf{supp}}
\newcommand\Eval{\ensuremath{\mathsf{Eval}}\xspace}
%%%

\newcommand{\ga}[1]{{ \textcolor{purple}{(Gorjan:  #1)}}{}}
\newcommand{\wf}[1]{{ \textcolor{orange}{(Bill:  #1)}}{}}

\bibliographystyle{plainnat}
\begin{document}


\title{Quantum obfuscation}
\author{Gorjan Alagic and Bill Fefferman}
\maketitle
\abstract{Encryption of data is fundamental to secure communication in the modern world. Beyond encryption of data lies \emph{obfuscation}, i.e., encryption of functionality. It has been known since 2001 that the most powerful means of obfuscating classical programs, so-called ``black-box obfuscation,'' is provably impossible. For years since, obfuscation was believed to always be either impossible or useless, depending on the particulars of its formal definition. However, several recent results have yielded candidate schemes that satisfy a definition weaker than black-box, and yet still have numerous applications.

In this work, we initialize the rigorous study of obfuscating programs \emph{via quantum-mechanical means.} We define notions of quantum obfuscation which encompass several natural variants. For instance, the input can describe classical or quantum functionality, and the output can be either a classical description or a quantum state. The obfuscator can also satisfy one of a number of obfuscation conditions: black-box, information-theoretic black-box, indistinguishability, and best-possible. We discuss a number of applications, including CPA-secure quantum encryption, quantum fully-homomorphic encryption, and quantum money. We then prove several impossibility results, extending a number of foundational papers on classical obfuscation to the quantum setting. In particular, we prove that quantum black-box obfuscation is impossible in a setting where adversaries can possess more than one output of the obfuscator (possibly even on the same input.) We also show that statistical indistinguishability obfuscation which outputs circuits or states is impossible, up to an unlikely complexity-theoretic collapse. Our proofs involve a new tool: chosen-ciphertext-secure encryption of quantum data, which we show is possible provided that quantum-secure one-way functions exist. 

We emphasize that our results leave open one intriguing possibility: obfuscating a classical or quantum circuit into a single, uncloneable quantum state. This indicates that, in spite of our results, quantum obfuscation may be significantly more powerful than its classical counterpart.

\newpage
\tableofcontents
\newpage


%%%%%%%%%%%%%%%%%%%%%%
\section{Introduction}\label{sec:intro}
%%%%%%%%%%%%%%%%%%%%%%

The ability to encrypt data is central to modern communications. In our daily lives, we frequently make use of a number of powerful tasks in encryption, such as private-key encryption, key exchange, and public-key encryption; in the near future, this may even include fully homomorphic encryption. Arguably the most powerful cryptographic ability is \emph{obfuscation}, which enables \emph{encryption of functionality}. Obfuscation implies (with some caveats) the ability to perform almost any other cryptographic task; this includes everything listed above, and much more. 

To understand obfuscation, it is useful to think about an obvious application: protecting intellectual property in software. In this setting, a software developer wishes to distribute their software to end users. However, the code contains a number of trade secrets which the developer does not want to become public. In order to accomplish this, the software is first passed through an obfuscator, and then published. The obfuscator must thus be an efficient algorithm that satisfies three core properties: 
\begin{enumerate}\label{def:obf-informal}
\item \emph{functional equivalence:} the input/output functionality does not change;
\item \emph{polynomial slowdown:} if the input program is efficient, then the output program is efficient;
\item \emph{obfuscation:} the code of the output program is ``hard to understand.''
\end{enumerate}
The last condition can be formulated rigorously in a number of ways. One possibility is the so-called ``virtual black-box'' condition, which says that the obfuscated program is no more useful than an impenetrable box which simply accepts inputs and produces outputs. While this condition appears to be too strong in the classical world, there are other formulations as well, with varying levels of strength and usefulness.

The study of encrypting classical data and classical programs is significantly complicated by the advent of \emph{quantum computation.} One widely-known complication is that certain data encryption schemes are no longer secure in the presence of quantum adversaries. The same may hold for obfuscation schemes. On the other hand, quantum mechanics may also enable us to perform cryptographic tasks that are impossible classically. It is thus natural to ask what quantum computation means for obfuscation of programs. In particular, we would like to answer the following questions:

\begin{itemize}
\item is it possible to quantumly obfuscate classical programs?
\item is it possible to obfuscate quantum programs?
\item how should we formulate quantum obfuscation in a rigorous manner?
\item which of the classical results about obfuscation carry over to the quantum setting?
\item can one use quantum mechanics to obfuscate in ways that are impossible classically?
\item are there interesting applications of any of the above?
\end{itemize}

We remark that, in order to address the above questions, we must also properly address the question of encrypting quantum data---a strictly simpler task than encrypting functionality. While information-theoretic encryption of quantum data has been considered before, in this setting we are interested in encryption of quantum data \emph{with computational assumptions}\footnote{Note that information-theoretic obfuscation is impossible if the adversary can execute the obfuscated program multiple times: a computationally unbounded adversary can then simply evaluate the program on all possible inputs, and use this to learn everything there is to know about the program.}. This latter subject has not yet received significant attention in literature.

Before continuing, we draw attention to the distinction between obfuscating \emph{programs} and obfuscating \emph{circuits}. While these two forms of obfuscation are closely related, there are some important technical differences. In this work, as in most theoretical works on obfuscation, we will focus on obfuscation of circuits. We view the circuit model as more convenient; it also tends to be preferred in the theoretical literature on both cryptography and quantum computation.

\subsection{Background}
%%%%%%%%%%%%%%%%%%%%%%

We now briefly review the current state of affairs in research on obfuscation. The classical case has been studied significantly. Quantum obfuscation, on the other hand, has received little to no attention.

\subsubsection{Classical obfuscation} 
%%%%

Ad-hoc obfuscation of software has been a fairly common practice for some time. In fact, simply compiling a program can be viewed as a form of obfuscation.  The earliest mention of obfuscation in the modern study of theoretical cryptography appears to be in the famous paper of Diffie and Hellman~\cite{DH76}. There, it was suggested that public-key cryptosystems might be constructible via obfuscation of private-key schemes; this was viewed as a reasonable possibility because writing code in an obfuscated manner seems relatively easy in practice. 

Arguably, the first major result in classical obfuscation was the 2001 proof by Barak et al. that virtual black-box obfuscation is impossible~\cite{BGIRSVY01, BGIRSVY12}. Their definition is based on the \emph{simulation paradigm}. More precisely, the obfuscation condition (i.e., the third condition in the \expreft{previous section}{def:obf-informal}) states that any efficient algorithm with access to an obfuscated circuit should be simulable by another efficient algorithm with only oracle (i.e., black-box) access to the original functionality. This definition is very natural in the setting of the aforementioned ``software intellectual property protection'' application: the end user can only learn that which is learnable by simply running the program. Barak et al. proved that  there exist circuit families which are unobfuscatable under this definition. They also showed that some of the most sought-after applications of black-box obfuscation are impossible. For instance, they showed that private-key encryption schemes cannot be transformed to public-key ones by obfuscating the encryption circuits in a generic manner.

The years following the Barak et al. result saw some limited progress in theoretical obfuscation. It was proved possible for some limited forms of functionality~\cite{CD08, Wee05}, and some additional limits were placed, e.g., on black-box obfuscation with auxiliary input~\cite{GK05}. An important step in formulating feasible notions of obfuscation was taken by Goldwasser and Rothblum; they defined \emph{indistinguishability obfuscation} and \emph{best-possible obfuscation}~\cite{GR07}. Both of these definitions alter the obfuscation condition, while leaving the functional-equivalence and polynomial-slowdown conditions unchanged. Under indistinguishability, it is required that the obfuscator maps functionally-equivalent circuits to indistinguishable distributions. Under best-possible, the obfuscator maps any circuit to a circuit from which the end user can ``learn the least.'' Both definitions have a perfect, statistical, and computational variant. Goldwasser and Rothblum proved that the two definitions are equivalent, and that the perfect and statistical versions are impossible (unless the PH collapses)~\cite{GR07}. This left one possibility: computational indistinguishability obfuscation. It was widely believed that computational indistinguishability was too weak of a condition to provide any interesting applications.

In 2013, in a breakthrough result, Garg et al. proposed a convincing candidate for computational indistinguishability obfuscation~\cite{GGHRSW13}. They proposed an obfuscation scheme for NC1 circuits, based on the presumed hardness of a problem in multilinear maps; they also showed how to use fully-homomorphic encryption (with NC1 decryption circuits) to ``bootstrap'' their NC1 scheme to obfuscation for all circuits. Around the same time, another breakthrough by Sahai and Waters showed how to use a computational indistinguishability obfuscator to achieve a wide-range of applications, via a new ``punctured programs'' technique~\cite{SW14}. These applications include chosen-ciphertext-secure public-key encryption, injective trapdoor functions, and oblivious transfer. Sahai and Waters suggested that the applications were so wide-ranging that indistinguishability obfuscation might become a ``'central hub' for cryptography''~\cite{SW14}. These two breakthroughs were followed by a flurry of new activity in the area, including several new proposals and applications~\cite{BGKPS14, BCCGKPR14, BZ14, BR14, GGHW14, HSW14}.

\subsubsection{Quantum obfuscation} 
%%%%

Quantum obfuscation is essentially an unexplored topic, and the present work appears to be the first rigorous treatment of the foundational questions. The question of whether quantum obfuscation is possible was posed as one of Scott Aaronson's ``semi-grand challenges'' for quantum computation~\cite{Aar05}. Since so little work on quantum obfuscation has appeared, our brief discussion will also mention some results that we believe are related. 

In~\cite{Aar09}, Aaronson proposed two relevant results. The first was a \emph{complexity-theoretic no-cloning theorem}, stating that cloning an unknown, random state by means of a black-box ``reflection oracle'' requires exponentially many queries. The second theorem stated that an oracle exists relative to which ``software copy-protection'' is possible. Unfortunately, a full version of~\cite{Aar09} with proofs never appeared, although the complexity-theoretic no-cloning theorem was eventually proved in a paper on quantum money~\cite{AC12}. In related work, Mosca and Stebila proposed a black-box quantum money scheme, and suggested the possibility of using a quantum circuit obfuscator in place of the black box~\cite{MS10}.

More recently, Alagic, Jeffery and Jordan proposed obfuscators for both classical (reversible) circuits and quantum circuits, based on ideas from topological quantum computation~\cite{ASS14}. The proposed obfuscator compiles the circuits into braids using certain high-dimensional representations of the braid group, and then applies an algorithm for putting braids into normal form. Although it is efficient, this algorithm does not satisfy any of the aforementioned obfuscation definitions; instead, it satisfies indistinguishability for a restricted set of circuit equivalences. The usefulness of such an obfuscator is unclear at this time.


\subsection{Summary of results}
%%%%%%%%%%%%%%%%%%%%%%

In this section, we summarize our results and discussions. These are divided by subject, with quantum encryption covered in \expref{Section}{sec:encryption}, quantum black-box obfuscation in \expref{Section}{sec:black-box}, and quantum indistinguishability obfuscation in \expref{Section}{sec:indistinguishability}.

\subsubsection{Quantum encryption}
%%%%

For us, \emph{quantum encryption} will mean the encryption of quantum states under computational assumptions. In this work, the crucial advantages of this form of quantum encryption over its information-theoretic analogues (e.g., the quantum one-time pad) are (i.) reusability of the key, and (ii.) chosen-ciphertext security. The results on quantum encryption which we will present are summarized below, and will be necessary in order to establish some of our results about black-box obfuscation. A complete treatment will appear in~\cite{ABFGSS15}.
\begin{enumerate}
\item \textbf{Quantum encryption schemes.} We define a notion of symmetric-key encryption scheme for quantum states, with reusable keys; these schemes consist of three quantum algorithms (key generation, encryption, and decryption) which satisfy correctness: under a fixed key, encryption followed by decryption must be equivalent to the identity.
\item \textbf{Chosen-ciphertext security for quantum encryption.} We define a notion of IND-CCA1 (or \emph{indistinguishability of ciphertexts under non-adaptive chosen ciphertext attacks}) for these schemes; this formalizes the idea of a ``lunchtime attack,'' where an adversary has complete access to all aspects of the encryption except the key itself, and is tasked with decrypting a challenge ciphertext later (presumably after lunch.)
\item \textbf{An IND-CCA1-secure construction.} We give a construction for an IND-CCA1-secure symmetric-key encryption scheme for quantum states, under the assumption that quantum-secure one-way functions (qOWF) exist. These are deterministic classical functions which are easy to compute, but hard to invert for quantum adversaries.
\end{enumerate}
We remark that, in contemporaneous work, Broadbent and Jeffrey also considered IND-CPA-secure public-key and symmetric-key quantum encryption; in addition, they considered partial quantum homomorphism~\cite{BJ15}. 

\subsubsection{Quantum black-box obfuscation}
%%%%

\paragraph{Definitions.} Our main results concern definitions, applications, and (im)possibility of quantum obfuscation in the virtual black-box setting. We will begin by defining the following.

\begin{enumerate}
\item \textbf{Quantum black-box obfuscator.} This is a polynomial-time quantum algorithm $\algo O$ which accepts quantum circuits $C$ as input, and produces quantum states $\algo O(C)$ as output. It preserves functionality, in the sense that there is a publicly known way to use $\algo O(C)$ and any input state $\ket{\psi}$ to produce the state $C \ket{\psi}$\,. It satisfies a black-box condition, which states that for polynomial-time quantum algorithms, possession of $\algo O(C)$ can be simulated by black-box access to $C$. This definition is a natural analogue of the classical black-box definition given in~\cite{BGIRSVY12}.
\item \textbf{Quantum ``two-circuit'' black-box obfuscator.} This obfuscator is precisely as above, except the obfuscation condition is strengthened to hold over arbitrary \emph{pairs of circuits} $(C_1, C_2)$. For us, this definition will be primarily useful because of its role in establishing certain impossibility results.
\item \textbf{Information-theoretic quantum black-box obfuscator.} This is a modification of the above definition, in which we posit that \emph{any} adversary with access to $\algo O(C)$ can be simulated by a \emph{polynomial-time} quantum simulator with black-box access to $C$. This definition is impossible classically, for obvious reasons: if $\algo O(C)$ is a classical state, then it can be reused an arbitrary number of times, enabling unbounded adversaries to discover everything about $C$.
\end{enumerate}

\paragraph{Impossibility.} We prove three impossibility results, which place several important restrictions on quantum obfuscation. Our impossibility proofs are based on the ideas of Barak et al.~\cite{BGIRSVY12}, with several important quantum adaptations, and a new quantum ingredient: the aforementioned IND-CCA1 quantum encryption.
\begin{enumerate}
\item \textbf{Two-state black-box obfuscation is impossible.} We prove that there exist families of circuit pairs which can reveal a secret if one is in possession of a circuit description for both of them, but not if one only has black-box access. This impossibility persists even if the obfuscation output is a quantum state, as opposed to a circuit description. Unlike the other results, it is also true even if the obfuscated states are \emph{not reusable.}
\item \textbf{If qOWFs exist, then obfuscation with more than one output is impossible.} For this proof, we combine the pairs from the circuit families in the two-circuit impossibility proof in order to build a single unobfuscatable family. The ability to execute obfuscated states from this family \emph{on themselves} is crucial here, and has two requirements: (i.) access to more than one obfuscation, even if the obfuscations are quantum states, and (ii.) secure encryption, which in turn requires the existence of OWFs. This result applies both to both quantum black-box obfuscators (as in the first definition above) and the information-theoretic variant (as in the third definition above.) 
\item \textbf{Classical algorithms for quantum obfuscation are impossible.} This result is unconditional, and follows directly from the previous result and Application 1 below. It can be viewed as an extension of the original Barak et al. impossibility result to the case of quantum functionality and quantum adversaries.
\end{enumerate}

\paragraph{Applications.} We then move on to discuss potential applications of quantum black-box obfuscators. We emphasize that (with the exception of the first one), all of these applications are still possible \emph{in some form} in spite of the above impossibility result. We view this as a strong indication that quantum obfuscation should be studied further. While some of the applications are analogues of known classical applications (as outlined in \cite{BGIRSVY12},) the last is special to the quantum setting. We are certain that many other quantum-specific applications are possible, given the combined advantage of obfuscation and no-cloning.

\begin{enumerate}
\item \textbf{Quantum-secure one-way functions.} We show that, if there exists a classical probabilistic algorithm for quantum obfuscation, then quantum-secure one-way functions exist. The above impossibility result rules this out, but the implication is nonetheless interesting; for one, it enables the very proof of the impossibility result itself! The one-way functions are essentially the functions computed by the obfuscator (with fixed randomness) on circuits with a ``hidden output.'' We are unable to extend this application to the setting of efficient quantum algorithms for obfuscation. We leave this as an interesting open problem, and note its connection to developing foundational primitives for quantum encryption.
\item \textbf{IND-CPA-secure quantum encryption.} In this case, the obfuscation algorithm can be quantum; moreover, we do not demand the existence of one-way functions or any other primitive.
\item \textbf{qOWF imply IND-CPA public-key encryption.} This application combines IND-CCA1-secure private-key encryption (which follows from qOWFs) with obfuscation of the encryption circuits. The result is public-key encryption of quantum states without the need for trapdoor permutations (as is done in~\cite{ABFGSS15}.)
\item \textbf{qOWF imply IND-CPA quantum fully homomorphic encryption.} This application combines the previous application, together with obfuscation of a universal decrypt-compute-encrypt circuit. Depending on the properties of the obfuscator, it may also satisfy \emph{compactness} (the requirement that communication between client and server does not scale with the size of the computation.)
\item \textbf{Public-key quantum money}. Using circuit obfuscation to produce public-key quantum money was first proposed by Mosca and Stebila~\cite{MS10}, using a complexity-theoretic no-cloning theorem proposed by Aaronson~\cite{Aar09} and proved by Aaronson and Christiano~\cite{AC12}. We outline the ideas here, and discuss the new limitations placed by our results.
\end{enumerate}

We emphasize that all the above applications except quantum money also work for achieving \emph{classical functionality} from a quantum obfuscator; however, depending on the details of the obfuscator and the application, this may require quantum algorithms for encryption and decryption, or even quantum ciphertexts.


\subsubsection{Quantum indistinguishability obfuscation}
%%%%

Lastly, we consider an alternative formulation of obfuscation, motivated by the classical definitions of indistinguishability obfuscation and best-possible set down by Barak et al.~\cite{BGIRSVY12} and Goldwasser and Rothblum~\cite{GR07}. We establish quantum analogues of the central results in those classical papers. In this setting, rather than comparing the obfuscation of the circuit to that of a black-box, we compare it to the obfuscations of other, functionally-equivalent circuits. Starting with the new definitions, our results are as follows.

\begin{enumerate}
\item \textbf{Quantum indistinguishability obfuscator.} Just as in the black-box definition, this is a polynomial-time quantum algorithm $\algo O$ which accepts quantum circuits $C$ as input, and produces ``functionally-equivalent'' quantum states $\algo O(C)$ as output. The obfuscation condition now states that functionally equivalent circuits are mapped to \emph{indistinguishable} states. Based on the kind of indistinguishability deployed in the definition, there are three variants of an indistinguishability obfuscator: perfect, statistical, and computational.
\item \textbf{Quantum best-possible obfuscator.} This is an algorithm precisely as above, except for the obfuscation condition: it now states that $\algo O(C)$ is the state that ``leaks least,'' among all states which are ``functionally-equivalent'' to $C$. There are again three variants: perfect, statistical, and computational.
\item \textbf{Equivalence of definitions.} We prove that each of the three variants of quantum indistinguishability obfuscation is equivalent to the analogous variant of quantum best-possible obfuscation.
\item \textbf{Impossibility of perfect and statistical indistinguishability obfuscation.} We end with a quantum version of the main result of~\cite{GR07}: a proof that perfect and statistical quantum indistinguishability obfuscation is impossible, unless coQMA is contained in QSZK. We remark that an analogous containment in the classical setting (i.e., coMA $\subseteq$ SZK) would imply a collapse of the polynomial-time hierarchy to the second level. One consequence of this result is that extending the obfuscator proposed in~\cite{ASS14} to full indistinguishability is impossible.
\item \textbf{Application: witness encryption for QMA.} Motivated by an analogue discussed in~\cite{GGSW13, GGHRSW13}, we show that a quantum indistinguishability obfuscator enables witness encryption for QMA. A witness encryption scheme for a language $L$ in QMA encrypts plaintexts $x$ using a particular instance $l$. The security condition states that, if $l \in L$, then a valid witness $w$ for $l \in L$ allows decryption; on the other hand, if $l \notin L$, then ciphertexts are indistinguishable. While witness encryption has several applications classically~\cite{GGSW13}, the quantum analogue has not been considered previously.
\end{enumerate}

We remark that, in the classical setting, indistinguishability obfuscation also implies functional encryption~\cite{GGHRSW13} and many more applications through the very successful ``punctured programs'' technique developed by Sahai and Waters~\cite{SW14}. We suspect that these results can also be adapted to the quantum setting, but leave them open for now.

%%%%%%%%%%%%%%%%%%%%%%
\section{Preliminaries}
%%%%%%%%%%%%%%%%%%%%%%

In this section, we set down some notation and basic terminology which we will use throughout the rest of the paper.

We will assume that the state space of a classical device can be identified with sets of bitstrings, i.e., $\{0, 1\}^n$ for some positive integer $n$. The notation $x \inrand \{0, 1\}$ will mean that $x$ is an $n$-bit string selected uniformly at random. The set of all bitstrings (of arbitrary length) will be denoted by $\{0, 1\}^*$. Classical functions will then be maps $f : \{0, 1\}^n \rightarrow \{0, 1\}^m$ from one set of bitstrings to another. We will also sometimes consider function families, written $f : \{0, 1\}^* \rightarrow \{0, 1\}^*$; these can be thought of as a function family $\{f_n\}_{n>0}$ indexed by the input size $n$. 

A classical circuit $C$ is a sequence of local logical gates which, when composed together, implement some (in general irreversible) function $f_C: \{0, 1\}^n \rightarrow \{0, 1\}^m$. The input size of $C$ is $n$, the output size is $m$, and the number of gates is denoted $|C|$. A probabilistic circuit is also a circuit, but with the input bits divided into two registers: the input register, and the ``coin'' register. A normal execution of a probabilistic circuit involves initializing the coin register with completely random bits, and inserting the input into the input register. We will frequently discuss ensembles of circuits; these are infinite families $\{C_n\}_{n > 0}$ of circuits, one for each possible input size. We will also sometimes make use of \emph{distributions of circuit ensembles}; these are infinite families $\mathcal C = \{\mathcal C_n\}_{n > 0}$ where each $\mathcal C_n$ is a finite family of circuits of input size $n$, along with a probability distribution $P_{\mathcal C, n}$. For a bitstring $x$, the notation $\mathcal C(x)$ will then denote the probability distribution (on bitstrings) resulting from running a random circuit from the family $\mathcal C_{|x|}$, selected according the distribution $P_{\mathcal C, |x|}$.

A deterministic classical algorithm $\algo A$ is simply a circuit ensemble. Running $\algo A$ on an input bitstring $x$ involves selecting the circuit with the appropriate input size, and executing it with input $x$. If the circuit ensemble is polynomial-time uniform (i.e., there exists a polynomial-time Turing machine that produces them), we will say that $\algo A$ is efficient; more precisely, it is then a classical deterministic polynomial-time algorithm, or PT for short. A probabilistic algorithm $\algo A'$ is an algorithm whose circuits are probabilistic. Running $\algo A'$ on an input bitstring $x$ involves selecting the circuit with the appropriate input size, initializing its coin register with uniformly random bits, and then executing it with input $x$. If the circuits of $\algo A'$ have size polynomial in their input size, we say that $\algo A'$ is an efficient, or classical probabilistic polynomial-time algorithm (PPT for short.) We will frequently use PPTs to model the most general efficient classical algorithms.

For our purposes, the space of pure states of a quantum device will be identified with a Hilbert space $\mathcal H_n \cong (\C_2)^{\otimes n}$ of a finite number $n$ of qubits. We will identify some fixed orthonormal basis (called the \emph{computational basis}) of $\mathcal H_n$ with the corresponding space $\{0, 1\}^n$ of classical states, so that, e.g, $\ket{x}$ for $x \in \{0, 1\}^n$ denotes a basis element of $\mathcal H_n$. The space of density operators of $n$ qubits will be denoted $\states (\mathcal H_n)$; a state in this space can be interpreted as a probabilistic mixture of pure states, albeit not in a unique way. We will discuss valid quantum transformations of three types. The first are measurements, which act on a state $\ket{\psi} \in \mathcal H_n$ by projecting some or all of the qubits into the computational basis states $\{\ket{0}, \ket{1}\}$. The second are unitary maps, i.e., linear operators $U: \mathcal H_n \rightarrow \mathcal H_n$ satisfying $U^\dagger U = \one_n$, where $\one_n$ denotes the $n$-qubit identity operator. The third are CPTP maps, i.e., completely positive trace-preserving maps $\Phi : \states (\mathcal H_n) \rightarrow \states (\mathcal H_m)$. CPTP maps are the most general type of evolution, encompassing unitary maps, measurement, and discarding (or tracing out) of subspaces. For example, a unitary operator $U \in U(\mathcal H_n)$ can be expressed as a CPTP map by writing $\rho \mapsto U\rho U^\dagger$, where $\rho \in \states (\mathcal H_n)$.

A quantum circuit $C$ is a sequence of local unitary gates on a fixed number (say $n$) of qubits; these gates , when composed together, implement some unitary operator $U_C \in U(2^n)$. Definitions of circuit ensembles and distributions over circuit ensembles are defined precisely as in the classical case. A quantum algorithm $\mathcal A$ is a (classically-)polynomial-time uniform family of quantum circuits; algorithms can also include measurements and discarding (or tracing-out) of subsystems, so long as these also admit efficient classical descriptions. The input and output size of an quantum algorithm can vary, and will have to be deduced from context.

%%%%%%%%%%%%%%%%%%%%%%
\section{Quantum encryption}\label{sec:encryption}
%%%%%%%%%%%%%%%%%%%%%%

In this section, we discuss a notion of encryption for quantum states with computational assumptions.  Interestingly, this topic has not received significant attention as yet. In \expref{Section}{sec:pseudo}, we will recall how to construct a classical function which appears pseudorandom to quantum adversaries, by means of a function which is one-way against quantum adversaries. In \expref{Section}{sec:scheme}, we define a notion of symmetric-key quantum encryption, together with associated notions of IND-CPA and IND-CCA1 security. We then describe a scheme which is IND-CCA1-secure under the assumption that quantum-secure one-way functions exist. While this particular scheme is new, encryption of quantum states with computational assumptions was also recently (and independently) considered by Broadbent and Jeffrey~\cite{BJ15}. A complete framework, including considerations about semantic security, will appear in an upcoming work~\cite{ABFGSS15}.

\subsection{Quantum-secure pseudorandomness}\label{sec:pseudo}
%%%%%%%%%%%%%%%%%%%%%%

We begin with two primitives for encryption: quantum-secure one-way functions, and quantum-secure pseudorandom functions. These are both classical, efficiently computable functions which are in some sense resistant to quantum analysis. In the case of one-way functions, we demand that inversion is hard; in the case of pseudorandom functions, we demand that distinguishing from perfectly random functions is hard.

\begin{definition}\label{def:quantum-secure-owf}
A PT-computable function $f:\{0,1\}^* \rightarrow \{0, 1\}^*$ is a quantum-secure one-way function (qOWF) if for every QPT $\algo A$, 
$$
\emph{Pr}_{x \inrand \{0, 1\}^n} \left[\algo A (f(x), 1^n) \in f^{-1}(f(x)) \right] \leq \negl(n)\,,
$$
where the probability is taken over $x \inrand \{0, 1\}^n$ as well as the measurements of $\algo A$.
\end{definition}

\begin{definition}\label{def:quantum-secure-prf}
A PT-computable function family $f_k : \{0,1\}^n \rightarrow \{0, 1\}^m$ is a quantum-secure pseudorandom function (qPRF) if for every QPT $\algo A$, 
$$
\left|\prob_{k \inrand \{0, 1\}^n} [\algo A^{f_k}(1^n) = 1] - \prob_{g \inrand \mathcal F_{n, m}}[ \algo A^g(1^n) = 1]\right|
\leq \negl(n)\,,
$$
where $\mathcal F_{n, m}$ denotes the space of all functions from $\{0,1\}^n$ to $\{0,1\}^m$.
\end{definition}

Classically, one-way functions are the fundamental primitive underpinning encryption. A series of basic results shows that one-way functions can be turned into pseudorandom functions, which can then be used for defining probabilistic encryption schemes. This series of results carries over to the quantum-secure case without much of a change (although some proofs are somewhat more involved.) For example, it is known how to construct qPRFs from qOWFs.

\begin{theorem}\label{thm:qOWF-implies-qPRF}
If quantum-secure one-way functions exist, then so do quantum-secure pseudorandom functions.
\end{theorem}
\begin{proof} (Sketch.) It is folklore that the well-known H{\aa}stad et al. result that pseudorandom generators can be constructed from any one-way function~\cite{HILL99} carries over to the quantum-secure case. Roughly speaking, the reasoning is that the reduction in the proof is done in a ``black-box'' way, i.e., only by feeding inputs into the adversary and then analyzing the resulting outputs. The quantum-secure case then simply involves replacing PPTs with QPTs in the appropriate places. Proving that the standard GGM construction~\cite{GGM86} of PRFs from pseudorandom generators is still secure in the setting of quantum adversaries is more involved; this was established by Zhandry~\cite{Zhandry2012}.
\end{proof}

\subsection{Symmetric-key encryption of quantum states}\label{sec:scheme}
%%%%%%%%%%%%%%%%%%%%%%

It is well-known how to encrypt quantum states with information-theoretic security, via the so-called quantum one-time pad. To encrypt a single-qubit state $\rho$, we choose two classical bits at random, use them to select a random Pauli matrix $P \in \{\one, X, Y, Z\}$, and perform $\rho \mapsto P \rho P^\dagger$. To encrypt an $n$-qubit quantum state $\rho$, we select $r \inrand \{0,1\}^{2n}$ and apply
\begin{equation}\label{eq:quantum-one-time-pad}
\rho \longmapsto P_r \rho P_r^\dagger\,,
\end{equation}
where $P_r$ denotes the element of the $n$-qubit Pauli group indexed by $r$. 

One disadvantage of the quantum one-time pad is that parties must share two bits of randomness for every qubit which they wish to transmit securely. In particular, one cannot securely exchange multiple messages with the same key. To address this issue, we must settle for computational security assumptions and use pseudorandomness to select $r$. A general encryption scheme for quantum states is then defined as follows.

\begin{definition}\label{def:encryption-scheme}
A symmetric-key quantum encryption scheme is a triple of QPTs:
\begin{itemize}
\item (key generation) $\KeyGen : 1^n \longmapsto k \in \{0, 1\}^n$;
\item (encryption) $\Enc_k : \states (\mathcal H_m) \longrightarrow \states (\mathcal H_c)$;
\item (decryption) $\Dec_k : \states (\mathcal H_c) \longrightarrow \states (\mathcal H_m)$;
\end{itemize}
where $m$ and $c$ are polynomial functions of $n$, and the QPTs satisfy $\| \Dec_k \circ \Enc_k - \one_m \|_\diamond \leq \negl(n)$ for all $k \in \emph{\supp}\,\KeyGen(1^n)$.
\end{definition}

Public-key quantum encryption schemes are defined in an analogous manner. The encryption schemes we will need must produce ciphertexts which are computationally indistinguishable. In some cases, the ciphertexts will need to remain indistinguishable even to adversaries which possess oracle access to the encryption algorithm (and sometimes also even the decryption algorithm.) This security notion is captured by the following definition.

\begin{definition}\label{def:IND}
A symmetric-key quantum encryption scheme is IND-secure if for all QPTs $\algo A, \algo A'$,
$$
\left|\prob[ (\algo A' \circ \Enc_k \otimes \one_s \circ \algo A) \cdot 1^n = 1] -
\prob[ (\algo A' \circ \Xi_{\Enc_k \ket{0^m}\bra{0^m}} \otimes \one_s \circ \algo A) \cdot 1^n = 1] \right|
\leq \negl(n)\,,
$$ 
where $\Xi_\sigma: \rho \mapsto \sigma$ is the ``forgetful'' map, and $s$ is a polynomial function of $n$. If $\algo A$ and $\algo A'$ have oracle access to $\Enc_k$, then we say that the scheme is IND-CPA secure. If in addition $\algo A'$ has oracle access to $\Dec_k$, then we say that the scheme is IND-CCA1 secure.
\end{definition}

The two QPTs $\algo A$ and $\algo A'$ together model the adversary. The definition above captures the idea of a certain ``security game'' between an adversary and a challenger. The game proceeds in steps: (i.) the key is selected and the adversary receives access to the appropriate oracles, (ii.) after some computation, the adversary transmits the first part of a bipartite state $\rho_{ms}$ to a challenger, (iii.) the challenger either encrypts this or replaces it with the encryption of $\ket{0^m}\bra{0^m}$, and then returns the result to the adversary, and (iv.) the adversary must decide which choice the challenger made. The scheme is considered secure if the adversary can do no better than random guessing. As shown in~\cite{ABFGSS15}, this definition is equivalent to a security notion called \emph{semantic security}; roughly speaking, this notion captures the idea that anyone that tries to compute anything about a plaintext gains no advantage by possessing its encryption. In addition, \expref{Definition}{def:IND} is equivalent to several natural variants, where e.g., the challenger chooses to encrypt one of two messages provided by the adversary, or where the game is played over multiple rounds. The latter guarantees security of transmitting multiple ciphertexts produced via encryption with the same key.

We now show how to use qPRFs to construct simple symmetric-key quantum encryption schemes that satisfy all of the above security conditions.

\begin{theorem}\label{thm:IND-CCA1}
If quantum-secure pseudorandom functions exist, then so do IND-CCA1-secure symmetric-key quantum encryption schemes.
\end{theorem}
\begin{proof}
Let $\{f_k\}$ be a qPRF. For simplicity we assume that each $f_k$ is a map from $\{0, 1\}^n$ to $\{0, 1\}^{2n}$. Recall that for $r \in \{0, 1\}^2n$, $P_r$ denotes the element of the $n$-qubit Pauli group indexed by $r$. Consider the following scheme:
\begin{itemize}
\item $\KeyGen(1^n)$: output $k \inrand \{0, 1\}^n$;
\item $\Enc_k(\rho)$: choose $r \inrand \{0, 1\}^n$; output $\ket{r}\bra{r} \otimes P_{f_k(r)} \rho P_{f_k(r)}^{\dagger}$;
\item $\Dec_k(\ket{r}\bra{r} \otimes \sigma)$: output  $P_{f_k(r)}^\dagger \rho P_{f_k(r)}$\,.
\end{itemize}
In the decryption algorithm, we may assume that the first register is always measured prior to decrypting. Correctness of the scheme is straightforward to check: decrypting with the same key and randomness simply undoes the Pauli operation.

We now sketch the proof that the scheme is IND-CCA1 secure; a complete proof will appear in \cite{ABFGSS15}. The key observation is that each query to the encryption oracle is no more useful than receiving a pair $(r, f_k(r))$ for $r \inrand \{0, 1\}^{2n}$, and that each decryption oracle is no more useful than receiving a pair $(r, f_k(r))$ for a string $r$ of the adversary's choice. Thus the adversary learns at most a polynomial number of values of $f_k$. Now, if $f_k$ is a perfectly random function, then these values are completely uncorrelated to the one used to encrypt the challenge. The scheme is thus secure simply by the information-theoretic security of the quantum one-time pad. On the other hand, if $f_k$ is a function in a qPRF, \expref{Definition}{def:quantum-secure-prf} guarantees oracle indistinguishability from perfectly random functions. It follows that, if $(\algo A, \algo A')$ can break the actual scheme, then by computational indistinguishability they would also break the perfect scheme, which is impossible.
\end{proof}

We emphasize that the above proof shows that, even in the case where the adversary chooses the randomness $r$ used by the $\Enc_k$ and $\Dec_k$ oracles, the scheme remains secure. Of course, the randomness for the challenge encryption must still be selected by the challenger. Finally, by combining \expref{Theorem}{thm:qOWF-implies-qPRF} and \expref{Theorem}{thm:IND-CCA1}, we have the following.

\begin{theorem}\label{thm:qOWF-implies-qSKE}
If quantum-secure one-way functions exist, then so do IND-CCA1-secure symmetric-key quantum encryption schemes.
\end{theorem}

%%%%%%%%%%%%%%%%%%%%%%
\section{Quantum black-box obfuscation}\label{sec:black-box}
%%%%%%%%%%%%%%%%%%%%%%

In this section, we discuss the virtual black-box framework for obfuscating quantum computations. We begin in \expref{Section}{sec:vbb-definitions} with a definition of black-box quantum obfuscator, motivated both by the classical analogue and an intuitive notion of what a ``good obfuscator'' should achieve. In \expref{Section}{sec:vbb-applications}, we outline several interesting cryptographic consequences that would follow from the existence of such an obfuscator. Finally, in \expref{Section}{vbb:impossibility}, we prove a few impossibility results which restrict the range of possibilities for the existence of black-box quantum obfuscators. Interestingly, our results leave open some possibilities, which include (restricted versions) of the most interesting applications. Indeed, it is conceivable that quantum obfuscation could be significantly more powerful than its classical counterpart.

\subsection{Definitions}\label{sec:vbb-definitions}
%%%%%%%%%%%%%%%%%%%%%%

Any reasonable notion of obfuscation involves giving the obfuscated circuit $\algo O(C)$ to an untrusted party. We accept as fundamental the idea that this obfuscated circuit should implement some particular, chosen functionality $f_C$, and that the object $\algo O(C)$ allows the untrusted party to execute that functionality. In the black-box formulation of obfuscation, we demand that this is effectively all that the untrusted party will ever be able to do. The rigorous formulation uses the simulation paradigm: anything which can be efficiently learned from the obfuscated circuit, should also be efficiently learnable simply by evaluating $f_C$ some polynomial number of times. This ``virtual black-box'' notion was first formulated by Barak et al.~\cite{BGIRSVY12}, and proved impossible to satisfy generically in the classical case.

In the quantum case, there are several complications. First, we are considering the obfuscation of quantum functionalities. This implies that the end user (and hence also any adversary) should be in possession of a quantum computer, and likewise for the simulator. Second, it is conceivable that the obfuscation may not just be another quantum circuit, which is simply a classical state describing a quantum computation. The obfuscator might instead output a quantum state, which is then to be employed by the end user to execute the desired functionality in some well-specified manner. These considerations motivate the following definition.
\begin{definition}\label{def:vbb-obfuscator}
A \textbf{black-box quantum obfuscator} is a pair of QPTs $(\algo J, \algo O)$ such that whenever $C$ is a polynomial-size $n$-qubit quantum circuit, the output of $\algo O$ is an $m$-qubit state $\algo O(C)$ satisfying
\begin{enumerate}
\item (polynomial expansion) $m = \text{poly}(n)$;
\item (functional equivalence) $\bigl\| \algo J ( \algo O(C) \otimes \rho ) - U_C \rho U_C^\dagger \bigr\|_\emph{tr} \leq \negl(n)$ for all $\rho \in \states(\mathcal H_n);$
\item (virtual black-box) for every QPT $\mathcal A$ there exists a QPT $\mathcal S^{U_C}$ such that
$$
\Bigl| \emph{Pr}\bigl[\mathcal A(\mathcal O(C)) = 1\bigr] - \emph{Pr}\bigl[\mathcal S^{U_C}\bigl(\ket{0^n}\bigr) = 1\bigr] \Bigr| \leq \negl(n)\,.
$$
\end{enumerate}
\end{definition}
We remark that one could consider variants where the ``interpreter'' algorithm $\algo J$ is fixed once and for all, or where $\algo O(C)$ itself consists of both a quantum ``advice state'' and a circuit which the end user should execute on the advice state and the desired input. It is straightforward to show that all of these variants are equivalent, in the sense that a black-box quantum obfuscator of each variant exists if and only if the other variants exist. Since we are primarily concerned with possibility vs impossibility, we will stick with the formulation in \expref{Definition}{def:vbb-obfuscator}.\\

\noindent\ga{Insert more careful version (with ensembles and distributions) of \expref{Definition}{def:vbb-obfuscator} here.}\\

Finally, we point out that the no-cloning theorem opens up the possibility of \emph{computationally unbounded adversaries.} In the classical case, such an adversary could simply execute the circuit on every input, and thus learn far more than is possible for a polynomial-time black-box simulator. Quantumly, however, a computationally unbounded adversary is restricted both by the no-cloning theorem and the limitations of measurement. The adversary may not be able to acquire multiple copies of the obfuscated state, and the single state may be partially (or completely) destroyed when measured. It is thus not \emph{a priori} clear that an unbounded adversary could always outmatch a polynomial-time black-box simulator. The appropriate definition is a straightforward modification of \expref{Definition}{def:vbb-obfuscator}, where we replace the third condition with the following:

\begin{enumerate}
\setcounter{enumi}{2}
\item \emph{(information-theoretic virtual black-box) for every quantum adversary $\mathcal A$ there exists a QPT $\mathcal S^{U_C}$ such that}
$$
\Bigl| \prob \bigl[\mathcal A(\mathcal O(C)) = 1\bigr] - \prob \bigl[\mathcal S^{U_C}\bigl(\ket{0^n}\bigr) = 1\bigr] \Bigr| \leq \negl(n)\,.
$$
\end{enumerate}




\noindent \ga{Note that our two-circuit impossibility proof holds even for these kinds of obfuscators, for a simple reason: there's already a QPT adversary that no QPT simulator can beat.}\\

\noindent \ga{Somewhere in here we need to mention that, when using obfuscated states, we will frequently write things like $\mathcal O(C) \ket{\varphi}$, which has the obvious meaning, but technically stands for appropriately using the interpreter (or the circuit given by the obfuscator), together with the advice state, as prescribed by the definition.}\\

\noindent \ga{Do we want to discuss inefficient obfuscators? I guess we can show that inefficient perfect indistinguishability obfuscators exist... and that these are black-box for any circuits that *do* have black-box obfuscations...}\\

\subsection{Applications}\label{sec:vbb-applications}
%%%%%%%%%%%%%%%%%%%%%%

In this section, we motivate the study of quantum black-box obfuscation by giving a few example applications. Many of these are motivated by known classical applications of classical black-box obfuscators. Although our impossibility results will put some restrictions on these applications, they remain interesting. In fact, some of the applications (such as quantum-secure one-way functions) will be used in the impossibility proofs themselves. We point out that, while most of the applications below are written in terms of quantum functionality (e.g., encryption of quantum states), one can just as well consider the weaker case of classical functionality, in this case achieved via quantum means (e.g., via a quantum algorithm for obfuscation.)

\subsubsection{Quantum-secure one-way functions}
%%%

The first application shows that, if there exists a classical algorithm for obfuscating quantum computations, then quantum-secure one-way functions exist. By the results discussed in \expref{Section}{sec:encryption}, this also implies the existence of quantum-secure pseudorandom generators, quantum-secure pseudorandom functions, and IND-CCA1-secure symmetric-key quantum encryption schemes.

\begin{prop}
If there exists a classical probabilistic algorithm which is a quantum black-box obfuscator, then quantum-secure one-way functions exist.
\end{prop}
\begin{proof}
The proof is essentially the same as that of Lemma 3.8 in \cite{BGIRSVY12}. For all $a \in \{0, 1\}^n$ and $b \in \{0, 1\}$, we define 
$$
U_{a, b} : \ket{x,\, y} \longmapsto
\begin{cases}
\ket{a,\, y \oplus b} &\text{ if } x = a;\\
\ket{x,\, y} &\text{ otherwise}.
\end{cases}
$$
Define a function $f : \{0, 1\}^* \rightarrow \{0, 1\}^*$ by $f(a, b, r) = \algo O_r(U_{a, b})$ where $\algo O$ is the obfuscator\footnote{For simplicity of notation, we omit $\algo J$ and assume that $f(a, b, r) = \algo O_r(U_{a, b})$ is in fact a classical circuit for $U_{a, b}$.} as in the hypothesis, and $\algo O_r$ denotes the same algorithm, but with randomness coins initialized to $r$. Clearly, inverting $f$ requires computing $b$ from $\algo O_r(U_{a, b})$. Moreover, with only black-box access to $U_{a, b}$ (for uniformly random $a, b$) the probability of correctly outputing $b$ in polynomial time is at most $1/2 + \negl(n)$.  By the black-box property of $\mathcal O$, we then have
\begin{align*}
\prob_{a, b} [ A(f(a, b, r)) = b] 
&= \prob_{a, b} [ A(\mathcal O_r(a, b)) = b ]\\
&\leq \prob_{a, b} \left[ S^{U_{a, b}}(1^n) = b\right] + \negl(n)\\
&\leq \frac{1}{2} + \negl(n)\,,
\end{align*}
which completes the proof.
\end{proof}

We remark that the above proof fails if the obfuscator is a quantum algorithm---even if its output is itself classical. The issue is that one-way functions must be deterministic; while one can turn a classical probabilistic algorithm into a deterministic one by making the coins part of the input, this is not possible quantumly. We leave the problem of constructing cryptographically useful primitives from a fully quantum obfuscator (or even just from a quantum encryption scheme) as an interesting open question. 

\subsubsection{CPA-secure private-key quantum encryption}
%%%

Can we say anything about encryption of data if we know that \emph{quantum} algorithms for quantum black-box obfuscation exist? While we do not know how to extract one-way functions, we can nonetheless produce useful encryption schemes, as follows.

\begin{prop}\label{prop:PKE-from-SKE}
If quantum black-box obfuscators exist, then so do IND-CPA-secure symmetric-key quantum encryption schemes.
\end{prop}
\begin{proof} (Sketch.)
Let $(\algo O, \algo J)$ be a quantum black-box obfuscator. We consider an adaptation of the unitary operator $U_{a, b}$ defined above, but now with Pauli group action instead of XOR, and with two $n$-bit registers:
$$
U'_{r, k} : \ket{x,\, y} \longmapsto
\begin{cases}
\ket{x,\, P_r^\dagger y} &\text{ if } x = k;\\
\ket{x,\, y} &\text{ otherwise},
\end{cases}
$$
Now consider the following scheme for encrypting $n$-qubit quantum states.
\begin{itemize}
\item $\KeyGen(1^n)$: output $k \inrand \{0, 1\}^n$;
\item $\Enc_{k}(\rho)$: choose $r \inrand \{0, 1\}^n$; output $P_r \rho P_r^\dagger \otimes \algo O(U_{r, k})$;
\item $\Dec_{k}(\sigma \otimes \tau)$: output  the second register of $\algo J(\tau \otimes \ket{k}\bra{k} \otimes \sigma)$.
\end{itemize}
To check correctness, we apply the functionality-preserving property of the obfuscator. A decryption of a valid encryption with the same key yields
\begin{align*}
\Dec_k (\Enc_k (\rho))  
&= \tr_1 \left[\algo J(\algo O(U_{r, k}) \otimes \ket{k}\bra{k} \otimes P_r \rho P_r^\dagger)\right] \\
&= \tr_1 \left[U_{r, k} (\ket{k}\bra{k} \otimes P_r \rho P_r^\dagger) U_{r, k}^\dagger\right] \\
&= \tr_1 \left[\ket{k}\bra{k} \otimes \rho \right] \\
& = \rho\,.
\end{align*}
as desired. IND-CPA security follows from the black-box property of the obfuscator, as follows. Let $\algo A$ be an adversary with access to the encryption oracle. Since the output of the encryption is a product state, $\algo A$ can be simulated by an adversary $\algo S$ that has only the first register of the ciphertext (i.e., $P_r \rho P_r^\dagger$) and black-box access to the unitary $U'_{r, k}$. It's then clear that $\algo S$ can only succeed in the challenge stage of \expref{Definition}{def:IND} by discovering the secret input for $U'_{r, k}$ or by guessing the response to the challenge. In any case, $\algo S$ (and hence also $\algo A$) succeeds with probability at most $1/2 + \negl(n)$.
\end{proof}

\subsubsection{Public-key encryption from private-key encryption}
%%%

As we now show, combining black-box obfuscation with one-way functions yields even stronger encryption functionality.

\begin{prop}\label{prop:PKE}
If quantum black-box obfuscators and quantum-secure one-way functions exist, then so do IND-CPA-secure public-key quantum encryption schemes.
\end{prop}
\begin{proof} (Sketch.)
Under the hypothesis, \expref{Theorem}{thm:qOWF-implies-qSKE} implies the existence of IND-CCA1-secure symmetric-key encryption schemes for quantum states. Let $(\KeyGen, \Enc, \Dec)$ be such a scheme; for concreteness, we may take the scheme described in \expref{Theorem}{thm:IND-CCA1}. For $x \in \{0, 1\}^n$, let $\Enc_{(x)}$ denote the encryption circuit for key $x$; this is the circuit that accepts two input registers (one for randomness, and one for the plaintext) and outputs the ciphertext. Now define a public-key encryption scheme $(\KeyGen', \Enc', \Dec')$ as follows.
\begin{itemize}
\item $\KeyGen'(1^n)$: output $sk : = k \inrand \{0, 1\}^n$ (secret key) and $pk := \algo O\left(\Enc_{(sk)}\right)$ (public key);
\item $\Enc'_{pk}(\rho)$: choose $r \inrand \{0, 1\}^n$; output $pk( \ket{r}\bra{r} \otimes \rho)$;
\item $\Dec'_{sk}(\sigma)$: output  $\Dec_{sk}(\sigma)$\,.
\end{itemize}
The correctness of this scheme follows directly from the functionality-preserving property of $\algo O$ and the correctness of the private-key scheme. To prove IND-CPA security for the public-key scheme, we rely on the black-box property. It implies that any QPT adversary $\algo A$ with access to the public key can be simulated by a QPT $\algo S$ having only black-box access to $\Enc_{(sk)}$. The QPT $\algo S$, in turn, can be simulated by a QPT $\algo S'$ which has both decryption and encryption oracles for the private-key scheme $(\KeyGen, \Enc, \Dec)$. It may not be immediately obvious that the decryption oracle is necessary; this is the case because black-box access to $\Enc_{(sk)}$ enables $\algo S$ to select the randomness used for encryption, thus gaining the ability to evaluate pairs $(r, f_{sk}(r))$ where $f$ is the qPRF from the private-key scheme. 

Now we have that, if $\algo A$ can distinguish ciphertexts during the challenge, then so can $\algo S'$; since the ciphertexts themselves are the same for the public-key scheme and the private-key scheme, this contradicts the IND-CCA1 security of the private-key scheme.
\end{proof}

A few remarks are in order. First, in~\cite{ABFGSS15} it is shown that IND-CPA-secure public-key quantum encryption schemes exist under the assumption that quantum-secure trapdoor permutations exist. This is a stronger assumption than one-way functions. \expref{Proposition}{prop:PKE} can then be thought of as replacing this strengthening of assumptions with an obfuscator. In~\cite{BJ15} it is shown how to use quantum-secure classical public-key encryption to produce quantum public-key encryption (by encrypting the key for the quantum one-time pad); this amounts to the same assumption on primitives as in~\cite{ABFGSS15}. An important difference between~\cite{ABFGSS15, BJ15} and \expref{Proposition}{prop:PKE} is that the scheme from \expref{Proposition}{prop:PKE} may have public keys which are quantum states. Such schemes have not been considered before, and (due to no-cloning) would have significantly different features from their classical counterparts.

An interesting question is if there could be public-key encryption for classical data with classical ciphertexts, but where the encryption procedure is performed by a quantum algorithm. While this question remains open, our impossibility results will show that this cannot be achieved in a generic way via \expref{Proposition}{prop:PKE}.

\subsubsection{Quantum fully homomorphic encryption}
%%%

We briefly recall the idea of fully homomorphic encryption (FHE). For thorough definitions and the appropriate notions of security in the fully quantum case, see~\cite{BJ15}. Without considering all of the details, we will view QFHE as an encryption scheme (just as in \expref{Definition}{def:encryption-scheme}), but where $\KeyGen$ produces an extra ``evaluation'' key $k_{\opn{eval}}$, and there is an ``evaluation'' algorithm:
\begin{itemize}
\item $\Eval_{k_{\opn{eval}}} : \states (\mathcal H_m \otimes \mathcal H_g) \longrightarrow \states (\mathcal H_m)$.
\end{itemize}
We imagine a party (henceforth, \emph{server}) in possession of $k_{\opn{eval}}$ and a ciphertext $\Enc_k(\rho)$ provided by another party (henceforth, \emph{client}.) The evaluation algorithm then enables the server to produce the ciphertext $\Enc_k(G_k \rho G_k^\dagger)$, where $G$ is a gate of the server's choice. A classical string describing the choice of gate $G$ (and which qubits $k, k+1, \dots$ of $\rho$ it should be applied to) is input into the register $\mathcal H_g$. In general, we may consider the case where $k_{\opn{eval}}$ is itself a quantum state. Depending on the details of the scheme, this key may be partly or fully consumed by $\Eval$; indeed, this is the case in~\cite{BJ15}. Depending on the consumption rate, this might violate the (classically standard) \emph{compactness} requirement for FHE, namely that the amount of communication between the client and the server should scale only with the size of the ciphertext, and not with the size of the computation the server wishes to perform.

\begin{prop}\label{prop:QFHE}
If quantum black-box obfuscators and one-way functions exist, then so do IND-CPA-secure quantum fully homomorphic encryption schemes.
\end{prop}
\begin{proof} (Sketch.)
We will consider the public-key case, which turns out to be simpler. Let $(\algo O, \algo J)$ be a quantum obfuscator, and $(\KeyGen, \Enc, \Dec)$ an IND-CPA-secure public-key scheme. We adapt $\KeyGen$ to produce an evaluation key, and describe the evaluation algorithm. We will require a universal circuit $U_\mu$ for performing gates on $m$-qubit states; this circuit accepts two inputs: an $m$-qubit state, and a description of a gate and indices of the qubits to which the gate should be applied. In our usage, $m$ will be the number of qubits of the ciphertext state.
\begin{itemize}
\item $\KeyGen'(1^n)$: output $\KeyGen(1^n) = (sk, pk)$ and $k_{\opn{eval}} = \mathcal O(\Enc_{pk} \circ U_\mu \circ \Dec_{sk})$;
\item $\Eval_{k_{\opn{eval}}} : \rho \otimes \ket{G}\bra{G} \longmapsto \algo J (k_{\opn{eval}} \otimes \rho \otimes \ket{G}\bra{G})$.
\end{itemize}
where $\ket{G}\bra{G}$ is again just a classical string instructing $U_\mu$ to apply the desired gate. A circuit for $\Enc_{pk} \circ U_\mu \circ \Dec_{sk}$ is given below; the gate register is represented by the bottom wire.\\ 
$$
\Qcircuit @C=1em @R=1.0em {
&\gate{\Dec_{sk}} 	&\multigate{1}{~~U_\mu~~} 	&\gate{\Enc_{pk}} 	&\qw\\
&\qw 			&\ghost{~~U_\mu~~}		&\qw 			&\qw\\ \\
}
$$
We now want to show that $(\KeyGen', \Enc, \Dec, \Eval)$ is a public-key QFHE scheme. The homomorphic property follows directly from the definition of $\Eval$ and the functionality-preserving property of the obfuscator. The security of the encryption scheme follows from IND-CPA security of $(\KeyGen, \Enc, \Dec)$ and the black-box property of $(\algo O, \algo J)$. The black-box property implies that each execution of the $\Eval$ algorithm is no more useful than providing the server with an encryption of $G\rho G^\dagger$. However, in the IND-CPA setting, the adversary can already use the CPA oracle to produce encryptions of \emph{arbitrary} plaintexts of her choice (as opposed to just ones which are modifications of the plaintext provided by the client.) There is one additional wrinkle: by repeatedly applying gates (or even just the identity), the adversary can also produce multiple encryptions during the challenge round. However, as shown in~\cite{BJ15}, single-message IND-CPA is equivalent to multiple-message IND-CPA. By the assumption that $(\KeyGen, \Enc, \Dec)$ is IND-CPA secure, it follows that the homomorphic scheme is also secure.

We remark that, in general, the encryption procedure $\Enc_{pk}$ may require an external source of randomness. This is certainly the case in classical encryption, but may not be required if the $\Enc$ algorithm is allowed to perform measurements. In any case, since we are starting with an IND-CPA public-key scheme, the adversary already has access to the public key and the ability to encrypt with randomness of her choice; the ability to choose randomness in $\Eval$ is of no additional benefit.
\end{proof}

\subsubsection{Public-key quantum money}
%%%

\paragraph{Quantum money.}
The idea of ``quantum money'' first arose in work by Wiesner~\cite{Wie1983}. The core idea is simple: use a quantum state for representing currency in such a way that the no-cloning theorem of quantum mechanics prevents counterfeiting. These ideas were refined and developed further in several works~\cite{Aar09, AC12, BBBW83, FGHLS12, MS10}; some of these works also included explicit proposals based on various hardness assumptions. 

Informally, a \emph{quantum money scheme} consists of two algorithms: \Mint, which produces quantum states, and \Verify, which accepts an input state and then either accepts or rejects. If the different states produced by \Mint are distinguishable, then we refer to them as \emph{bills}; if they are indistinguishable, then we call them \emph{tokens} (if \Verify consumes them) or \emph{coins} (if \Verify does not consume them.) In all quantum money schemes, we imagine an authority (typically called the bank) which runs \Mint repeatedly to produce money; in addition, the \Verify algorithm should accept only on states produced by the bank. Depending on the particular scheme, this might only be true if \Verify is executed by the bank (private-key money), or it might be true for any party (public-key money.)

In this language, Wiesner's original idea~\cite{Wie1983} was for a private-key scheme for bills, which is as follows. Each execution of \Mint produces two random classical bitstrings $r, s \in \{0, 1\}^{2n}$ as well as an $n$-qubit quantum state $\ket{\psi_r}$, with each qubit initialized in one of the states $\ket{0}, \ket{1}, \ket{+}, \ket{-}$, as determined by the bits of $r$. The bank records the pair $(r, s)$ in a secret table, and publishes $(s, \ket{\psi_r})$. The bank verifies by using $s$ to look up the correct $r$ in the table, and then performing the measurements in the correct basis and checking the results against $r$. 

\paragraph{Public-key money from circuit obfuscation.}
While private-key money schemes are relatively straightforward to construct, public-key proposals appear to be much more difficult, and require computational assumptions. In analogy to its role in producing public-key encryption schemes from private-key ones (\expref{Proposition}{prop:PKE-from-SKE}), an obfuscator can sometimes be used to turn private-key money schemes to public-key ones. The use of an obfuscator to create a particular quantum money scheme was considered by Mosca and Stebila~\cite{MS10}. Their scheme (in our language) is as follows. Each execution of \Mint produces a Haar-random $n$-qubit quantum state $\ket{\psi}$, together with the obfuscation $\mathcal O(U_\psi)$ of a circuit\footnote{For most $\ket{\psi}$, the circuit $U_\psi$ will not have polynomial length. However, as pointed out by~\cite{Aar09}, one can instead select $\ket{\psi}$ from an approximate $t$-design without a significant loss in security.} for $U_\psi = \one - 2\ket{\psi}\bra{\psi}$. The bill consists of the pair $(\mathcal O(U_\psi), \ket{\psi})$. $\Verify(\ket{\varphi})$ consists of executing the following:
$$
\Qcircuit @C=1em @R=1.0em {
\lstick{\ket{0}} &\gate{H} 	&\ctrl{1}				&\gate{H} 	&\qw &\meter\\
\lstick{\ket{\varphi}} &\qw 	&\gate{\algo O(U_\psi)}	&\qw 	&\qw &\qw\\ \\
}
$$
and accepting iff the measurement returns $1$. It's easy to check that the above succeeds only on valid states; moreover, in that case, the state $\ket{\psi}$ is output in the second register, so that verification can be repeated. To show resistance of the above scheme to counterfeiting, one can use Aaronson's Complexity-Theoretic No-Cloning Theorem~\cite{Aar09}, which states that cloning the state $\ket{\psi}$ while in possession of oracle access to $\ket{U_\psi}$ requires $\Omega(2^{n/2})$ queries. The first published proof of this theorem (as well as its first appearance in the form required here) was in~\cite{AC12}. 

Unfortunately, we will later show that obfuscation of quantum circuits in the form required by Mosca and Stebila is impossible. What remains possible is a setting in which both $\ket{\psi}$ and $\mathcal O(U_\psi)$ are quantum states, and another circuit (which is publicly known and independent of $\ket{\psi}$) is used for verification. Moreover, as we will also show, any black-box obfuscation scheme which outputs states that can be efficiently cloned is also impossible. We thus conjecture the following.
\begin{conjecture}
If quantum black-box obfuscators exist, then so do public-key quantum money schemes.
\end{conjecture}
If the relevant obfuscation is a consumable state, then this would result in a token scheme. If it can be reused to perform verification repeatedly\footnote{For example, if successful verification also outputs another state which is sufficiently close to the original state.}, then the result would be a bills scheme. We remark that, in any case, all of the public-key money states discussed above should be authenticated by the bank; otherwise a merchant would only know that he was handed \emph{some} pair (state, circuit) where the circuit executed on the state outputs ``accept''---a clearly inadequate state of affairs.

\subsection{Impossibility results}\label{vbb:impossibility}
%%%%%%%%%%%%%%%%%%%%%%

\subsubsection{Impossibility of two-circuit obfuscation}\label{sec-twocircuit}
%%%
Barak et. al. \cite{BGIRSVY12} showed that black-box obfuscation is impossible by constructing an explicit circuit family that cannot be black-box obfuscated. 
Here we present a similar result, showing that quantum black-box obfuscation is impossible in any setting where the adversary can gain access to two outputs of the obfuscator on \emph{different} inputs. We formalize this notion by defining a ``black-box two-circuit obfuscator,''  defined just as in Definition \ref{def:vbb-obfuscator} but with the following strengthening of the virtual black-box condition:
\begin{enumerate}
\setcounter{enumi}{2}
\item \emph{(two-circuit virtual black-box) for every pair of quantum circuits $C_1$ and $C_2$ and every quantum adversary $\mathcal A$ there exists a quantum simulator $\mathcal S^{U_{C_1}, U_{C_2}}$ and a negligible $\epsilon_2$ such that}
$$
\Bigl| \text{Pr}[\mathcal A(\mathcal O(C_1) \otimes \mathcal O(C_2)) = 1] - \text{Pr}\bigl[\mathcal S^{U_{C_1}, U_{C_2}}\bigl(\ket{0}^{\otimes |C_1| + |C_2|}\bigr) = 1\bigr] \Bigr| \leq \epsilon_2(n, \min\{|C_1|,|C_2|\})\,.
$$
\end{enumerate}

We now show that there exists a family of circuits which is unobfuscatable under the above definition. We emphasize that our result holds even when the outputs of the obfuscator are quantum states, and even if these states are \emph{single-use only}, i.e., if the interpreter $\mathcal J$ irrevocably destroys the obfuscated state during use. 

We first define a \emph{circuit-pair family} to be an ensemble of distributions over pairs of circuits. More precisely, if $\mathcal C$ is a circuit-pair family, then there exists a Turing machine M which, on input a positive integer parameter $n$ (in unary), outputs a classical description of a pair of circuits $(C_n, D_n)$ drawn at random from some distribution $\mathcal C_n$ on pairs of poly$(n)$-size circuits. If $M$ is polynomial-time, then we say that $\mathcal C$ is a \emph{poly-time circuit-pair family.}

We also define a \emph{state-pair family} analogously. If $\mathcal C'$ is a state-pair family, then there exists a (not necessarily polynomial-time) quantum algorithm which, on input $n$ in unary, outputs a pair of density operators $(\rho_n, \sigma_n)$ drawn at random from some distribution $\mathcal C_n'$ on quantum states on poly$(n)$-many qubits. 

With these definitions, we can now state our first impossibility result.

\begin{theorem}
There exists a poly-time quantum circuit-pair family $\mathcal C$ such that no state-pair family is a two-circuit black-box obfuscation of $\mathcal C$.
\end{theorem}
\begin{proof}
Let $(\mathcal O, \mathcal J)$ be a black-box quantum two-circuit obfuscator. 
The poly-time quantum circuit-pair family $\mathcal C$ consists of quantum circuits for implementing the following pairs of unitary operators. Each pair is parameterized by an input size $n$, as well as bitstrings $a, b$ chosen uniformly at random from $\{0, 1\}^n$.
\begin{align}\label{eq:pair-impossible}
U_{a, b} &: \ket{x,\, y} \longmapsto
\begin{cases}
\ket{x,\, y \oplus b} &\text{ if } x = a;\\
\ket{x,\, y} &\text{ otherwise}.
\end{cases}\\
V_{a, b} &: \ket{C,\, z} \longmapsto
\begin{cases}
\ket{C,\, z \oplus 1} &\text{ if } C(a) = b;\\
\ket{C,\, z} &\text{ otherwise}.
\end{cases}
\end{align}
The registers indexed by $x$ and $y$ are of size $n$. The register indexed by $C$ accepts a circuit description (under some fixed encoding), and needs to be able to handle inputs of size $|\algo O(C_{a, b})|$ (i.e. of size equal to the number of qubits in the state $\algo O(C_{a, b})$). Here $C_{a, b}$ is a fixed, explicit poly$(n)$-size circuit for $U_{a, b}$. The second register of $V_{a, b}$ has size one. 

Note that both of these unitaries can be implemented by efficient quantum circuits. We choose some particular set of such circuits, and henceforth denote them by $C_{a,b}$ and $D_{a,b}$, respectively. The idea for the proof is as follows. Consider an adversary $\algo A$ which is ignorant of the randomly selected $a$ and $b$, and consider two scenarios: in the first, $\algo A$ is given access to \emph{any} pair of \emph{circuits} that implement $U_{a, b}$ and $V_{a, b}$; in the second, $\algo A$ only has oracle access to $U_{a, b}$ and $V_{a, b}$. The point is that, in the first case, $\algo A$ can execute $V_{a, b}$ on a circuit for $U_{a, b}$; provided that the latter is not too long, $\algo A$ will achieve something that is impossible to do with only black-box access. Specifically, it is only in the first case that $\algo A$ will be able to tell if the first circuit/oracle implements $U_{a, b}$, or if it has surreptitiously been replaced by the identity operator!

Things are somewhat complicated by the fact that the obfuscator outputs states instead of circuits. We will need to enable $\algo A$ to execute these states on one another. It will thus be necessary to replace $D_{a, b}$ with a related circuit $D_{a, b}'$. Roughly speaking, this circuit will check if its input, when interpreted as a quantum advice state to the algorithm $\algo J$, maps the input $a$ to the output $b$. A precise description follows. First, $D_{a, b}'$ will have three registers: an input register of $m$ qubits, a work register of $2n$ qubits, and an output register of $1$ qubit initialized in the $|0\rangle$ state. When given as input a quantum state $\rho$ on $m$ qubits, it will initialize the first $n$ bits of the work register to $|a\rangle$, then execute the appropriate unitary circuit of $\algo J$ on $\rho \otimes \ket{a}$. Finally, if the output register of the latter computation contains $\ket{b}$, $D_{a, b}'$ will flip the contents of the output register. We remark that, by a simple counting argument over circuits, this occurs for only an exponentially small fraction of possible input states $\rho$.

Recall that the $2n$-qubit identity operator is denoted by $\one_{2n}$, and is implemented by the obvious circuit which we will denote by $I_{2n}$. We observe that, for every QPT algorithm $\mathcal S$ there exists a polynomial $t$ and a negligible $\epsilon_1$ so that:

\begin{equation}\label{eqn11}
\Bigl| \text{Pr}\bigl[\mathcal S^{U_{a,b}, D_{a,b}'}\bigl(\ket{0}^{\otimes t(n)}\bigr) = 1\bigr]
- \text{Pr}\bigl[\mathcal S^{Id_{2n}, D_{a,b}'}\bigl(\ket{0}^{\otimes t(n)}\bigr) = 1\bigr] \Bigr| 
\leq \epsilon_1(n)\,.
\end{equation}
Here the probability is taken over the uniformly random choice of $a$ and $b$ as well as all of the measurement outcomes of $\algo S$. The above is an easy corollary of the tightness of the Grover bound for unstructured quantum search \cite{BBBV}. Indeed, given the definitions of $U_{a, b}$ and $D_{a, b}'$, it's clear that with only polynomial queries and no knowledge of $a$ or $b$, $\algo S$ is faced precisely with unstructured search for an exponentially small ``marked space.'' This marked space is only encountered if $\algo S$ correctly guesses $a$, or correctly guesses an obfuscation of a circuit that maps $a$ to $b$.
  
Now consider the QPT algorithm $\mathcal{A}$ that, given as input the obfuscated states $\mathcal{O}(C)$ and $\mathcal{O}(D)$, simply executes the quantum algorithm $\mathcal{J}$ on their tensor product, accepting if and only if the outcome is $1$. Notice that this succeeds with constant probability $\alpha > 0$ if $C$ is functionally equivalent to $C_{a,b}$ and $D$ is functionally equivalent to $D_{a,b}'$. On the other hand, this same algorithm $\algo A$ accepts with at most negligible probability when $C$ is functionally equivalent to $I_{2n}$ (and $D$ is still functionally equivalent to $D_{a, b}'$); indeed, this only happens is if $a = b$. Thus there exists a negligible function $\epsilon_2$ so that:

\begin{equation}\label{eqn2}
\Bigl| \text{Pr}\bigl[\mathcal A(\mathcal{O}(D'_{a,b}),\mathcal{O}(I_{2n})) = 1\bigr]
- \text{Pr}\bigl[\mathcal A\bigl(\mathcal{O}(D'_{a,b})\otimes\mathcal{O}(C_{a,b})\bigr) = 1\bigr] \Bigr| 
\geq \alpha-\epsilon_2(n)\,.
\end{equation}

To complete the proof, we explicitly define the poly-time circuit-pair family $\mathcal C$. The distribution $\mathcal C_n$ is generated by choosing $a, b$ uniformly at random from $\{0,1\}^n$, and then choosing a bit $r \in {0, 1}$ at random; if $r = 0$, we output the circuit pair $(C_{a,b}, D'_{a,b})$, and otherwise we output $(I_{2n}, D'_{a,b})$. For this distribution, equations \eqref{eqn11} and \eqref{eqn2} together show that no state-pair family is an obfuscation of $\mathcal C$.
\end{proof}

\subsubsection{Impossibility of black-box obfuscation}
%%%

\ga{stopped editing here.}

Our goal in this section is to modify the black-box quantum two-circuit impossibility proof, in the prior section, to demonstrate impossibility of quantum circuit obfuscation with cloneable outputs.

We first prove this in the simpler setting in which the output of the obfuscator is a quantum circuit.

\begin{definition}\label{def:blackboxcircuit}
A \textbf{black-box quantum obfuscator with circuit output} is a QPT $\algo O$ such that whenever $C$ is an $n$-qubit quantum circuit, the output is a quantum circuit $\algo O(C)$ so that:
\begin{enumerate}
\item (polynomial slowdown)  $|\algo O(C)| = \text{poly}(n, |C|)$;
\item (functional equivalence) $\bigl\| U_{\algo{O}(C)} - U_{C}\|_\diamond \leq \negl(n, |C|)$;
\item (virtual black-box) for every QPT adversary $\mathcal A$ there exists a QPT simulator $\mathcal S^{U_C}$ such that:
$$
\Bigl| \emph{Pr}[\mathcal A(\algo{O}(C)) = 1] - \emph{Pr}\bigl[\mathcal S^{U_C}\bigl(\ket{0}^{\otimes |C|}\bigr) = 1\bigr] \Bigr| \leq \negl(n, |C|)\,.
$$
\end{enumerate}
\end{definition}
%\begin{definition}\label{blackboxclone}
%A \textbf{black-box quantum obfuscator with cloneable outputs} is a pair of QPTs $(\algo J, \algo O)$ such that whenever $C$ is an $n$-qubit quantum circuit, the output $\mathcal{O}(C)$ is a $2m$-qubit state $\rho\otimes\rho$, where  $\rho$ satisfies:
%\begin{enumerate}
%\item (polynomial slowdown)  $m = \text{poly}(n, |C|)$;
%\item (functional equivalence) $\bigl\| \algo J ( \rho \otimes \,\cdot\, ) - C \,\cdot\, C^\dagger \bigr\|_\diamond \leq \negl(n, |C|)$;
%\item (virtual black-box) for every QPT adversary $\mathcal A$ there exists a QPT simulator $\mathcal S^{U_C}$ such that:
%$$
%\Bigl| \emph{Pr}[\mathcal A(\rho_{(i)}) = 1] - \emph{Pr}\bigl[\mathcal S^{U_C}\bigl(\ket{0}^{\otimes |C|}\bigr) = 1\bigr] \Bigr| \leq \negl(n, |C|)\,.
%$$
%\end{enumerate}
%\end{definition}

Our main impossibility result is a modification of the proof in the prior section, following the classical proof \cite{BGIRSVY12}.
\begin{theorem}
There exists an ensemble of distributions $\lbrace\mathcal{H}_n\rbrace_{n\in\mathbb{N}}$ over quantum circuits, $C_n$, of size poly(n), such that no ensemble of distributions over circuits is a black-box quantum obfuscation with circuit output, of $\lbrace\mathcal{H}_n\rbrace_{n\in\mathbb{N}}$.
\end{theorem}
\begin{proof}
This proof works by carefully extending the two-circuit construction.  Let $\mathcal O$ be a black-box quantum obfuscator with circuit outputs.
First we give a general definition which will be useful:
\begin{definition}We define the {\bf combined quantum circuit} of a finite collection of quantum circuits each with $n$ input qubits, $\lbrace C_1,C_2,...,C_k \rbrace$, to be the circuit that takes two registers, a control register of $\log{k}$ qubits, and an input register of $n$ qubits, and controlled on the value of the first register applies the respective quantum circuit to the input register.   
\end{definition}

Notice that if each circuit $C_i$ in the collection is polynomial size, and $k$ is bounded by a polynomial in $n$, then the associated combined quantum circuit is also polynomial sized.   

Now consider the two circuits  $C_{a,b}$ and $D_{a,b}$ from Section \ref{sec-twocircuit}, as well as the circuit $I_{2n}$, which simply implements the identity operator on $2n$ qubits.  Also consider the combined quantum circuits of $C_{a,b}$ and $D_{a,b}$ which we denote $C_{a,b}\#D_{a,b}$ and the combined quantum circuit of $I_{2n}$ and $D_{a,b}$ which we denote by $I_{2n}\#D_{a,b}$.  Using the reasoning of the argument from Section \ref{sec-twocircuit}, these combined quantum circuits are indistinguishable from the perspective of any QPT simulator that is given only black-box access.  On the other hand, unlike the prior proof, it is not immediately apparent that there exists an algorithm $\algo{A}$ that can distinguish inputs $\algo{O}(C_{a,b}\#D_{a,b})$ from $\algo{O}(I_{2n}\#D_{a,b})$.  This is because the naive algorithm that runs the obfuscation, $\algo{O}(C_{a,b}\#D_{a,b})$ on itself, to simulate the effect of running $D_{a,b}$ on $C_{a,b}$ does not work, since in general the size of $\algo{O}(C_{a,b}\#D_{a,b})$ will be  polynomially larger than the input length of the circuit $D_{a,b}$.

To fix this issue, following the construction in the classical impossibility proof \cite{BGIRSVY12}, our solution is to prove the following theorem about the existence of a distribution over circuits that allows a circuit of fixed input length to test whether a given quantum circuit $C$ of arbitrary polynomial size maps an input $a$ to an output $b$. In particular, we show:
\begin{lemma}\label{lemma-circuitdistribution}
	If quantum-secure one-way functions exist, then for each $n\in\mathbb{N}$ and $a,b\in\{0,1\}^n$ there exists a distribution $\mathcal{D}_{a,b}$ over circuits with the following properties:
\begin{enumerate}
\item{Every $D\in \supp(\mathcal{D}_{a,b})$ is a circuit of size $poly(n)$.  Furthermore, there exists a QPT algorithm that,  for every $n\in\mathbb{N}$ on input $a,b\in\{0,1\}^n$, samples the distribution $\mathcal{D}_{a,b}$.}
\item{There is a QPT algorithm $\algo{A}$ so that for all $n \in \mathbb{N}$,   $a,b\in\{0,1\}^n$ and $D\in \supp(\mathcal{D}_{a,b})$, and for every circuit $C$, if $C|a\rangle|0^n\rangle=|a\rangle|b\rangle$, then $\algo{A}^{U_D}(C,1^n)=a$.}
\item{For any QPT $S$, $\Pr[S^{U_D}(1^n)=a]\leq neg(n)$, where the probability is over $a,b\in\{0,1\}^n$, $D\sim \mathcal{D}_{a,b}$, and the measurement of $S$.}	
\end{enumerate}
\end{lemma}

\begin{proof}  We follow closely the proof of Lemma 3.6 from the classical impossibility result \cite{BGIRSVY12}, basically constructing a basic quantum private-key "homomorphic encryption" scheme.  We think of each circuit $D\in \supp(\mathcal{D}_{a,b})$ as the combined quantum circuit of the following three circuits which depend on a private key $K\in\{0,1\}^{2n}$ which will be used with the IND-CCA1-secure symmetric-key quantum encryption scheme from Theorem \ref{thm:IND-CCA1}.
\begin{enumerate}
\item{$E_{K,a}(r)$ takes a polynomial length input $r$, which we will think of as the source of randomness, and uses this randomness to output  $\Enc_{K}(|a\rangle)$}.  
\item{$\Homorcl_K(C,\rho,r)$ takes a quantum circuit $C$, a state $\rho$, and a polynomial length string $r$, which we will think of as the source of randomness, and uses this randomness to output $\Enc_K(C(\Dec_K(\rho)))$}.
\item{$B_{K,a,b}$ takes a quantum state $\rho$ and outputs $|a\rangle$, if $\Dec_{K}(\rho)=|b\rangle$, and otherwise outputs $|0^n\rangle$.}	
\end{enumerate}

Clearly given $a$ and $b$, $\mathcal{D}_{a,b}$ can be sampled efficiently by choosing $K$ uniformly at random and outputting the combined quantum circuit $E_{K,a}\#\Homorcl_{K}\#B_{K,a,b}$, establishing Property 1 from the Lemma.  Furthermore, notice that the QPT algorithm $\algo{A}$ that gets the description of a quantum circuit $C$ as input can check if $C(a)=b$ by using the three circuits comprising $D_{K,a,b}$ to simulate $C$ gate-by-gate, using $\Homorcl_K$ initialized on the output of the $E_{K,a}$ circuit, and finally outputs the value of the circuit $B_{K,a,b}$, establishing Property 2.  

It remains to verify Property 3, that no QPT simulator algorithm that has black-box access to  each of the three algorithms comprising $D_{K,a,b}$ can discover $a$ with non-negligible probability.  We'll need the following lemma:
\begin{lemma}\label{hybrid-lemma}
Let (Enc,Dec) be a IND-CCA1-secure symmetric-key quantum encryption scheme, and $\Homorcl$ be as in the prior discussion.  Then, for all $n$ qubit quantum states $\rho$ and every QPT algorithm $\algo{A}$:
$$
\Bigl| \emph{Pr}[\mathcal A^{\Homorcl_K,\Enc_K}(\Enc_K(|0^n\rangle)) = 1] - \emph{Pr}\bigl[\mathcal A^{\Homorcl_K,\Enc_K}(\Enc_K(\rho)) = 1 \Bigr| \leq \negl(n)\,.
$$
Where the probabilities are over $K$ chosen uniformly from $\{0,1\}^n$ and the measurement outcome of $\algo{A}$.
\end{lemma}
\begin{proof}
Assume there's an algorithm $\algo{A}$ that violates the claim.  We'll show that this would break the IND-CCA1 security of the quantum encryption scheme.

To do this we first argue that we can replace the responses to all of $\algo{A}$'s queries to the $\Homorcl_K$ oracle with Encryptions of $|0^n\rangle$, with only a negligible loss in $\algo{A}$'s distinguishing gap.  Consider the computation of $\algo{A}$ on input $\Enc_K(\rho)$ for each quantum state $\rho$ on $n$ qubits, and consider "hybrid" computations, where in the $i$-th hybrid, the first $i$ queries of $\algo{A}$ to the $\Homorcl_K$ oracle are answered using the $\Homorcl_K$ oracle and the rest are answered using $\Enc_K(|0^n\rangle)$.  Notice that any gap in distinguishing between the $i$ and $i+1$st hybrid must be due to the $i+1$st query $\algo{A}$ makes to $\Homorcl_K$, which is what differs between the hybrids. But we can now use this algorithm to create an adversary in violation of IND-CCA1 security of the encryption scheme.  In particular, consider the algorithm that uses the $\Enc_k$ and $\Dec_k$ oracles to simulate all calls to the $\Homorcl_K$ oracle before receiving the challenge ciphertext, uses the challenge ciphertext as our answer to the $i+1$st query to $\Homorcl_K$, and then answers all subsequent queries to $\Homorcl_K$ with $\Enc_K(|0^n\rangle)$.  Thus any gap between the $i$ and $i+1$st hybrid amounts to a distinguishing gap between quantum ciphertexts, in violation of IND-CCA1 security.

After this is established, we have that $\algo{A}$ can distinguish an encryption of $|0^n\rangle$ from an encryption of $\rho$, when given access to only an encryption oracle, again in violation of IND-CCA1. 
\end{proof}
Notice that Lemma \ref{hybrid-lemma} suffices to establish Property 3, since giving the simulator algorithm black-box access to the three unitaries that comprise $D_{a,b}$ is equivalent to giving $S$ black-box access to each circuit separately.  Notice that black-box access to  $E_{K,a}$ is no more powerful than giving it access to polynomially many queries of $\Enc_{K}$, and giving black-box access to $B_{K,a,b}$ does not allow $S$ to discover $a$ with more than negligible probability, since it returns $|0^n\rangle$ on all but an exponentially small fraction of the space.  Lemma \ref{hybrid-lemma} proves security in the presence of the $\Homorcl$ and $\Enc$ oracle.
\end{proof}

Now we are ready to adapt the two-circuit impossibility proof of Section \ref{sec-twocircuit} to the single circuit case.  First for given $a,b$ let the distribution $\mathcal{D}_{a,b}$ be the distribution over circuits constructed in Lemma \ref{lemma-circuitdistribution}.  Then consider the following two distributions over circuits: \begin{enumerate}
 	\item{$\mathcal{F}_{n}$: Choose $a,b$ uniformly at random from $\{0,1\}^n$,  sample a circuit $D$ from $\mathcal{D}_{a,b}$ and output $C_{a,b}\# D_{a,b}$ 
}
\item{$\mathcal{G}_{n}$: Choose $a,b$ uniformly at random from $\{0,1\}^n$,  sample a circuit $D$ from $\mathcal{D}_{a,b}$ and output $I_{2n}\# D_{a,b}$ }
 \end{enumerate}
 By Property 2 of Lemma \ref{lemma-circuitdistribution} 
 there exists an algorithm $\algo{A}$ that, on input $\mathcal{O}(C)$, accepts if $C$ was sampled from $\algo{F}_n$ and rejects if $C$ was sampled from $\algo{G}_n$.  Thus there exists a constant $\alpha$ and a negligible function $\epsilon_1$ so that:
$$\Bigl| \text{Pr}\bigl[\mathcal A(\mathcal{O}(\mathcal{F}_n)) = 1\bigr]- \text{Pr}\bigl[\mathcal A\bigl(\mathcal{O}(\mathcal{G}_n)\bigr) = 1\bigr] \Bigr| \geq \alpha-\epsilon_1(n)\,.$$
 
 While by Property 3 of Lemma \ref{lemma-circuitdistribution}, we know that for every QPT $S$ there exists some negligible function $\epsilon_2$ so that: 
 $$\label{eqn1}\Bigl| \text{Pr}\bigl[\mathcal S^{\mathcal{F}_n}\bigl(\ket{0}^{\otimes n}\bigr) = 1\bigr]- \text{Pr}\bigl[\mathcal S^{\mathcal{G}_n}\bigl(\ket{0}^{\otimes n}\bigr) = 1\bigr] \Bigr| \leq \epsilon_2(n)\,.$$
Notice that this contradicts the virtual black-box condition of Definition \ref{def:black-box}.
  
 
\end{proof}




%The main impossibility proof goes here. Outline:
%
%\begin{itemize}
%\item show how to paste the above pair of circuits to make a single circuit;
%\item running circuit on itself computes something a simulator cannot;
%\item but running circuits on themselves doesn't work, so we need to patch it with encryption...
%\end{itemize}
%This proof will now be divided into three parts.  In the first part we 
%
%


%\begin{definition}\label{blackboxclone}
%A \textbf{black-box quantum obfuscator with cloneable outputs} is a pair of QPTs $(\algo J, \algo O)$ such that whenever $C$ is an $n$-qubit quantum circuit, the output of $\algo O$ is a $2m$-qubit state $\algo O(C)\otimes \algo O(C)$ satisfying
%\begin{enumerate}
%\item (polynomial slowdown)  $m = \text{poly}(n, |C|)$;
%\item (functional equivalence) $\bigl\| \algo J ( \algo O(C) \otimes \,\cdot\, ) - C \,\cdot\, C^\dagger \bigr\|_\diamond \leq \negl(n, |C|)$;
%\item (virtual black-box) for every QPT adversary $\mathcal A$ there exists a QPT simulator $\mathcal S^{U_C}$ such that
%$$
%\Bigl| \emph{Pr}[\mathcal A(\mathcal O(C)) = 1] - \emph{Pr}\bigl[\mathcal S^{U_C}\bigl(\ket{0}^{\otimes |C|}\bigr) = 1\bigr] \Bigr| \leq \negl(n, |C|)\,.
%$$
%\end{enumerate}
%\end{definition}
%Our main impossibility result is, in analogue to the proof in the prior section and in \cite{BGIRSVY12}:
%\begin{theorem}
%There exists an ensemble of distributions $\lbrace\mathcal{H}_n\rbrace_{n\in\mathbb{N}}$ over quantum circuits, $C_n$, of size poly(n), such that no pair of quantum circuits is a black-box quantum obfuscation with cloneable outputs, of this ensemble of distributions.
%\end{theorem}
%\begin{proof}
%This proof works by carefully extending the two-circuit construction to show that a similar construction establishes impossibility for the cloneable case.  
%First we give a general definition which will be useful:
%\begin{definition}We define the {\bf combined quantum circuit} of a finite collection of quantum circuits each with $n$ input qubits, $\lbrace C_1,C_2,...,C_k \rbrace$, to be the circuit that takes two registers, a control register of $\log{k}$ qubits, and an input register of $n$ qubits, and controlled on the value of the first register applies the respective quantum circuit to the input register.   
%\end{definition}
%
%Notice that if each circuit $C_i$ in the collection is polynomial size, and $k$ is bounded by a polynomial in $n$, then the associated combined quantum circuit is also polynomial size. 
%
%Now consider the two unitaries  $U_{a,b}$ and $V_{a,b}$ from Section \ref{sec-twocircuit}, and their respective quantum circuits $C_{a,b}$ and $D_{a,b}$, as well as the circuit $Z_{2n}$, which simply implements the identity operator on $2n$ qubits.  Also consider the combined circuits of $C_{a,b}$ and $D_{a,b}$ which we denote $C_{a,b}\#D_{a,b}$ and the combined quantum circuit of $Z_{2n}$ and $D_{a,b}$ which we denote by $Z_{2n}\#D_{a,b}$.  Clearly, the     	
%\end{proof}





%The main impossibility proof goes here. Outline:
%
%\begin{itemize}
%\item show how to paste the above pair of circuits to make a single circuit;
%\item running circuit on itself computes something a simulator cannot;
%\item but running circuits on themselves doesn't work, so we need to patch it with encryption...
%\end{itemize}
%This proof will now be divided into three parts.  In the first part we 
%
%


%%%%%%%%%%%%%%%%%%%%%%
\section{Quantum indistinguishability obfuscation}\label{sec:indistinguishability}
%%%%%%%%%%%%%%%%%%%%%%

\subsection{Definitions}
%%%%%%%%%%%%%%%%%%%%%%

\begin{definition}\label{def:indistinguishability}
An \textbf{indistinguishability quantum obfuscator} is a pair $(\mathcal J, \mathcal O)$ where $\mathcal J$ is an interpreter and $\mathcal O$ is a quantum algorithm which on input an $n$-qubit quantum circuit $C$ outputs an $m$-qubit quantum state $\mathcal O(C)$, such that
\begin{enumerate}
\item (polynomial slowdown) $m = \text{poly}(n, |C|).$
\item (functional equivalence) there exists a negligible $\epsilon_1$ such that $\bigl\| \mathcal J_n^{\mathcal O(C)} - U_C \bigr\|_\diamond \leq \epsilon_1(n, |C|)$;
\item (indistinguishability) if a pair of circuits $C_1$ and $C_2$ satisfy $|C_1| = |C_2|$ and $\bigl\|U_{C_1} - U_{C_2}\bigr\|_\diamond \leq \epsilon_3(n, |C|)$, then $\bigl\|\mathcal O(C_1) - \mathcal O(C_2)\bigr\|_\emph{tr} \leq \epsilon_4(n, |C|)$.
\end{enumerate}
\end{definition}
\noindent As before, we will select $\epsilon_3$ and $\epsilon_4$ appropriately later. For a definition of best-possible obfuscation, we replace condition (3) above with the following:

\begin{enumerate}
\setcounter{enumi}{2}
\item \emph{(best-possible) for every pair of quantum circuits $C_1$ and $C_2$ that satisfy $|C_1| = |C_2|$ and $\bigl\|U_{C_1} - U_{C_2}\bigr\|_\diamond \leq \epsilon_3(n, |C|)$ and every quantum adversary $\mathcal A$, there exists a quantum simulator $\mathcal S$ and a negligible $\epsilon_2$ such that}
$$
\Bigl| \text{Pr}[\mathcal A(\mathcal O(C_1)) = 1] - \text{Pr}\bigl[\mathcal S\bigl(C_2\bigr) = 1\bigr] \Bigr| \leq \epsilon_2(n, |C|)\,.
$$
\end{enumerate}

The intuition behind the above definition is the following: any information $\mathcal A(\mathcal O(C_1))$ that is ``leaked'' by the obfuscation $\mathcal O(C_1)$ can actually be recovered from \emph{any} functionally equivalent, similarly-sized circuit $C_2$. In this sense, among all such circuits, the circuit $\mathcal O(C_1)$ is one that leaks the least. It's not hard to see that an efficient obfuscator satisfies the best-possible condition if and only if it satisfies the indistinguishability condition. This justifies \expref{Definition}{def:indistinguishability} as a natural choice.

\ga{To mention somewhere: GR07 observed that, if a circuit family \emph{has} a black-box obfuscation, then a computational indistinguishability obfuscator must compute it. So it's conceivable that many of the interesting black-box applications carry over to the quantum case. Of course, one could say that this is exactly why the recent classical results have worked.}


\subsection{Applications}
%%%%%%%%%%%%%%%%%%%%%%

\noindent {\bf Example: quantum witness encryption.} The classical idea of witness encryption comes from a paper of Garg, Gentry, Sahai and Waters \cite{GGSW13}, and its connection to indistinguishability obfuscation was first considered by Garg et. al. \cite{GGHRSW13}. In the quantum case, we set up the problem as follows. Suppose Alice wishes to encrypt a quantum state $\rho$, but not to a particular key or for a particular person; instead, the encryption is tied to a challenge question, and anyone that can answer the question correctly can decrypt the plaintext. 

More formally, we consider any QMA$_{1}$ language $L$, such as Quantum $3$SAT \cite{GN13}.  We would like Alice to be able to encrypt her state $\rho$ using a particular problem instance, in this case a Hamilonian $H=\sum_{i=1}^r \Pi_{i}$, which is the sum of 3-local projectors that act on an $n$ qubit Hilbert space.  If $H$ is a ``yes" instance, i.e., if there exists a ground state with energy 0, we'd like to allow Bob, who holds the ground state, to be able to decrypt Alice's message.  Likewise, if $H$ is a no instance, we demand that no QPT algorithm can distinguish between encryptions of any two quantum states  on the same number of qubits.  Interestingly, the definition says nothing about the case where $H$ is a yes instance but a ground state is not known. While this may seem counterintuitive, the classical primitive has a number of natural applications (e.g., public-key encryption and identity encryption, see \cite{GGSW13}). 

We now show that Witness Encryption for languages in $QMA_1$ is possible, assuming the existence of a quantum computational indistinguishability obfuscator.  Given an instance $H$ of quantum $3$SAT and a quantum state $\rho$, consider the quantum circuit $Q_{H,\rho}(\sigma)$, which outputs $\rho$ if $\sigma$ is the ground state of $H$ and else outputs the all $0$ basis state.  We claim the Indistinguishability Obfuscation $\algo{O}(Q_{H,\rho})$ is a valid witness encryption for Quantum $3$SAT and $\rho$.  Correctness of decryption is clear, since the ability to provide a valid ground state for $H$ allows Bob to use $\algo{O}(Q_{H,\rho})$ to obtain Alice's state $\rho$.  Of course, if $H$ does not have a ground state with energy $0$, no matter which state Bob provides, $Q_{H,\rho}$ outputs 0, and so by the definition of Indistinguishability Obfuscation, the obfuscation of $F_{H,\rho_1}$ will be computationally indistinguishable from the obfuscation of $F_{H,\rho_2}$ for any two quantum states $\rho_1,\rho_2$ on the same number of qubits.

%Alice outputs a ciphertext $F_\phi \ket x$ where $\phi$ is a quantum 3-SAT formula, such that there exists an efficient algorithm Eval with the property that Eval$(F_\phi\ket x, \ket y) = \ket x$ if $\ket y$ is a satisfying assignment for $\phi$. The security requirement is that if $\phi$ does not have a satisfying assignment, then the ensembles $F_\phi \ket x$ and $F_\phi \ket {x'}$ are quantum indistinguishable (formally, this now requires a definition of distinguishing \emph{quantum} ensembles) whenever $\ket x$ and $\ket{x'}$ are quantum states on the same number of qubits. 
%
%The problem of quantum witness encryption can be solved using a quantum best-possible obfuscator $\mathcal O$, as follows. First, Alice selects a random Clifford (or Pauli) circuit $C$. She then writes down a quantum circuit $M_C$ which accepts two registers (and some ancillas), such that $M \ket z \ket y \ket 0 = \ket {C^{-1}z} \ket y \ket 0$ when $\ket y$ is a satisfying assignment for $\phi$, and $M \ket z \ket{y'} \ket 0 = \ket z \ket{y'} \ket 0$ for $\ket{y'}$ not a satisfying assignment for $\phi$. The ciphertext $F_\phi \ket x$ will consist of the pair $(C \ket x, \mathcal O(M_C))$. A recipient with a satisfying assignment $\ket y$ can decrypt by computing $\mathcal O(M_C) \ket {Cx} \ket y \ket 0$. On the other hand, if no satisfying assignment exists, then $M_C$ acts like the identity operator on every input. By the definition of best-possible, a quantum adversary can learn nothing more from $\mathcal O(M_C)$ than she could from the trivial circuit with no gates. Moreover, by the design property of Cliffords (or Paulis) the adversary also observes $\ket{Cx}$ to be a maximally mixed state.

%\begin{itemize}
%\item Stephen has a description of how to build the circuit $M_C$, and that should be added.
%\item I guess the state $C \ket x$ and the circuit $M_C$ are correlated. Is this a problem? This probably has to be addressed by defining quantum indistinguishability of quantum ensembles, and then showing that quantum indistinguishability of the classical ensemble $\mathcal O(M_C)$ plus 2-design property on $C \ket{x}$ implies quantum indistinguishability of the quantum ensemble $(C \ket x, \mathcal O(M_C))$.
%\item what does $M_C$ do if you feed in a state that has a little bit of projection into a satisfying assignment? I guess that, unless the size of the projection is $1/$poly, it's still indistinguishable from identity...
%\item I have some ideas on why the above is exactly the right definition (e.g., weakening to $\phi$ being just a 3-SAT formula opens it up to being solved by classical obfuscation.) 
%\end{itemize}

\subsection{Equivalence of indistinguishability and best-possible}
%%%%%%%%%%%%%%%%%%%%%%

\ga{some old stuff below should be removed, but most still applies}

In what follows, for the sake of simplicity we omit the perfect, statistical, and classical variants of the definitions; one can arrive at these versions simply by replacing quantum indistinguishability of the relevant ensembles to one of the other notions. We will always be obfuscating quantum circuits, so when the word ``quantum'' appears in front of ``obfuscator'', this refers to the type of indistinguishability. We say that two uniform quantum circuit families $\mathcal C'$ and $\mathcal C''$ are equivalent if they consist of functionally equivalent circuits of the same size; more precisely, for every $n$, $|\mathcal C'_n| = |\mathcal C''_n| = 1$ and $|C'_n| = |C''_n|$  and $U_{C'_n} = U_{C''_n}$.

\begin{itemize}
\item the exact-same-length condition seems too strong, but it does appear in GR too, along with a later comment about how it can be removed. I guess some care is needed.
\end{itemize}

\begin{definition} A classical probabilistic algorithm $\mathcal O$ that takes as input a quantum circuit $C$ and outputs another quantum circuit $\mathcal O(C)$ is a quantum {\bf best-possible obfuscator} for the family $\mathcal C$ if it satisfies properties (1) and (2) from Definition \ref{def:black-box}, as well as the following property:
\begin{enumerate}
\setcounter{enumi}{2}
\item for any learner (uniform quantum circuit family) $\mathcal L$, there is a simulator (uniform quantum circuit family) $\mathcal S$ and a negligible $\phi$ such that, for all uniform equivalent subfamilies $\mathcal C', \mathcal C''$ of $\mathcal C$, the two ensembles $\mathcal L(\mathcal O(\mathcal C'))$ and $\mathcal S(\mathcal C'')$ are quantumly indistinguishable.
\end{enumerate}
\end{definition}


\ga{some old stuff below}

\begin{definition} A classical probabilistic algorithm $\mathcal O$ that takes as input a quantum circuit $C$ and outputs another quantum circuit $\mathcal O(C)$ is a quantum {\bf indistinguishability obfuscator} for the family $\mathcal C$ if it satisfies properties (1) and (2) from Definition \ref{def:black-box}, as well as the following property:
\begin{enumerate}
\setcounter{enumi}{2}
\item for all uniform equivalent subfamilies $\mathcal C', \mathcal C''$ of $\mathcal C$, the two ensembles $\mathcal O(\mathcal C')$ and $\mathcal O(\mathcal C'')$ are quantumly indistinguishable.
\end{enumerate}
\end{definition}

\begin{itemize}
\item in all of the above, we could have considered obfuscating quantum states, or even using quantum algorithms to obfuscate classical descriptions of a quantum circuit. Why is this the ``right'' case (or at least an interesting one)?
\end{itemize}

With the definitions set up as above, many of the proofs of Goldwasser and Rothblum go through with little to no changes.

\begin{proposition} There exists an inefficient perfect indistinguishability obfuscator for all quantum circuits.
\end{proposition}
\begin{proof}
The obfuscator just picks the lexicographically first circuit which implements the same unitary as the given circuit. Looping through lexicographically ordered circuits can be done in PSPACE, and equivalence-checking can be done in QMA $\subset$ QIP = PSPACE too.
\end{proof}

\begin{itemize}
\item what's the smallest class that one can do this in?
\end{itemize}

\begin{proposition} If $\mathcal O$ is a best-possible quantum obfuscator for a circuit family $\mathcal C$, then it is also a quantum indistinguishability obfuscator for $\mathcal C$.
\end{proposition}
\begin{proof}
Let $\mathcal C'$ and $\mathcal C''$ be uniform equivalent subfamilies of $\mathcal C$, and let $\mathcal L$ be the trivial learner that simply implements the identity operator. By the best-possible property, there is a simulator $\mathcal S$ such that $\mathcal S(\mathcal C'')$ is quantum indistinguishable from $\mathcal L ( \mathcal O( \mathcal C')) = \mathcal O ( \mathcal C')$. By the same property, we also have that $\mathcal S(\mathcal C'')$ is quantum indistinguishable from $\mathcal L(\mathcal O (\mathcal C'')) = \mathcal O( \mathcal C'')$. By the transitivity property of indistinguishability, it follows that $\mathcal O(\mathcal C')$ is indistinguishable from $\mathcal O(\mathcal C'')$.
\end{proof}

\begin{proposition} If $\mathcal O$ is an efficient quantum indistinguishability obfuscator for a circuit family $\mathcal C$, then it is also an efficient quantum best-possible obfuscator for $\mathcal C$.
\end{proposition}
\begin{proof}
Let $\mathcal C'$ and $\mathcal C''$ be equivalent subfamilies of $\mathcal C$, and let $\mathcal L$ be a (quantum) learner whose output on $\mathcal C'$ is the ensemble $\mathcal L(\mathcal O (\mathcal C'))$. We define a (quantum) simulator by setting $\mathcal S = \mathcal L \circ \mathcal O$; its output on $\mathcal C''$ is then the ensemble $\mathcal L(\mathcal O(\mathcal C''))$. Since the ensembles $\mathcal O(\mathcal C')$ and $\mathcal O(\mathcal C'')$ are quantum indistinguishable, so are their images under $\mathcal L$.
\end{proof}
\subsection{Impossibility of statistical obfuscators}
%%%%%%%%%%%%%%%%%%%%%%

Recall the following computational problems and corresponding completeness results.

\begin{definition} \emph{Identity Check.}\\
\indent Input: an $n$-qubit quantum circuit $C$ and parameters $a$,$b$ so that $b-a\geq 1/poly(n)$. \\
\indent Promise: $\min_\alpha \| U - e^{i \alpha} I \|$ is less than $a$ or greater than $b$.\\
\indent Output: YES in the former case and NO in the latter.
\end{definition}
\begin{theorem} 
The problem Identity Check is coQMA-complete \cite{JWB03}. 
\end{theorem}

Given an $m$-qubit state $\rho$, let $\tr_{(l, m)}[\rho]$ denote the result of tracing out qubits $l$ through $m$. Nothing is traced out if $l > m$.

\begin{definition} \emph{Quantum State Distinguishability}\\
\indent Input: $m$-qubit quantum circuits $C_1$ and $C_2$, positive integer $k \leq m$ and parameters $a,b$ such that $a<b^2$.\\
\indent Promise: let $\rho_i = \tr_{(k+1, m)}[C_i\ket{0^m}\bra{0^m}C_i^\dagger]$; then $\|\rho_0 - \rho_1\|_\emph{tr}$ is less than $a$ or greater than $b$.\\
\indent Output: YES in the former case and NO in the latter.
\end{definition}

\begin{theorem}
The problem Quantum State Distinguishability is QSZK-complete \cite{Watrous02}.
\end{theorem}

We will in fact only need the containment part of the above theorem.

\begin{theorem}
If there exists a polynomial-time indistinguishability quantum obfuscator, then coQMA is contained in QSZK.
\end{theorem}
\begin{proof}
We will actually show coQMA $\subset$ BQP$^{\text{QSZK}}$; since BQP is contained in QSZK, the result will follow. Let $a$ and $b$ satisfy $b-a = 1 / \text{poly}(n)$. We will solve Identity Check using a subroutine that solves Quantum State Distinguishability. 

Let $C$ be the input, i.e., a classical description of an $n$-qubit quantum circuit. Create an identity circuit $D$ with an equal number of inputs as $C$, and of equal length to $C$. Let $O_C$ be a circuit that initializes a register with the classical state $\ket{C}$ containing the classical description of $C$, and applies the circuit of $\mathcal O$ which corresponds to the input length $|C|$. Likewise, let $O_D$ be be a circuit that initializes a register with the classical state $\ket{D}$ containing the classical description of $D$, and applies the circuit of $\mathcal O$ which corresponds to the input length $|D| = |C|$. Note that, after tracing out ancillas, the outputs of these circuits are given by
$$
\tr_\text{anc.} \bigl[O_C\ket{0}\bra{0}O_C^\dagger\bigr] = \mathcal O(C)
\qquad \text{and} \qquad
\tr_\text{anc.} \bigl[O_D\ket{0}\bra{0}O_D^\dagger\bigr] = \mathcal O(D)\,.
$$
Now apply the subroutine for solving quantum state distinguishability to the pair $(O_C, O_D)$. If it says ``close'', we output YES; otherwise we output NO. Let's show that this has solved $(a, b)$-identity-check. Note that the states $\mathcal O(C)$ and $\mathcal O(D)$ must have the same number of qubits, and denote that number by $m$.
\begin{itemize}
\item \textbf{completness.} In this case, the obfuscated states satisfy $\|\mathcal O(C) - \mathcal O(D)\|_\text{tr} \leq \alpha$.  By the definition of the induced trace norm, this implies that $\|\mathcal J_{\mathcal O(C)}^n - \mathcal J_{\mathcal O(D)}^n\|_\diamond \leq \alpha$. By functional equivalence for $C$ and $D$ and the triangle inequality, it follows that $\|U_C - U_D\|_\diamond = \|U_C - I\|_\diamond \leq \alpha$, as desired.

\item \textbf{soundness.} In this case, the obfuscated states satisfy $\|\mathcal O(C) - \mathcal O(D)\|_\text{tr} \geq \beta$. We claim that this implies $\|U_C - U_D\|_\diamond > b$. Suppose this is not the case, i.e., that these operators are in fact close; then by the indistinguishability property, it would follow that $\mathcal O(C)$ and $\mathcal O(D)$ are close as well, a contradiction.
\end{itemize}
The above amounts to a BQP$^\text{QSZK}$ protocol for a coQMA-hard problem, thus placing coQMA in QSZK.
\end{proof}

%\subsection{Impossibility of statistical obfuscators}
%%%%%%%%%%%%%%%%%%%%%%%
%
%Recall the following computational problems and corresponding completeness results.
%
%\begin{definition} \emph{$(a, b)$-Identity Check.}\\
%\indent Input: an $n$-qubit quantum circuit $C$. \\
%\indent Promise: $\min_\alpha \| U - e^{i \alpha} I \|$ is less than $a$ or greater than $b$.\\
%\indent Output: YES in the former case and NO in the latter.
%\end{definition}
%\begin{theorem} 
%The problem $(a, b)$-Identity Check is coQMA-complete if $b - a \leq 1/\emph{poly}(n)$. 
%\end{theorem}
%\noindent \textbf{Note:} apparently Rosgen showed that distinguishing mixed state computations is QIP-complete. Does this mean that we could have even stronger impossibility results if we asked for obfuscators that could obfuscate quantum circuits that included measurements?
%
%Given an $m$-qubit state $\rho$, let $\tr_{(l, m)}[\rho]$ denote the result of tracing out qubits $l$ through $m$. Nothing is traced out if $l > m$.
%
%\begin{definition} \emph{$(a, b)$-Quantum State Distinguishability}\\
%\indent Input: $m$-qubit quantum circuits $C_1$ and $C_2$, positive integer $k \leq m$.\\
%\indent Promise: let $\rho_i = \tr_{(k+1, m)}[C_i\ket{0^m}\bra{0^m}C_i^\dagger]$; then $\|\rho_0 - \rho_1\|_\emph{tr}$ is less than $a$ or greater than $b$.\\
%\indent Output: YES in the former case and NO in the latter.
%\end{definition}
%
%\begin{theorem}
%The problem $(a, b)$-Quantum State Distinguishability is QSZK-complete if $a < b^2$.
%\end{theorem}
%
%We will in fact only need the containment part of the above theorem.
%
%\begin{theorem}
%If there exists a polynomial-time indistinguishability quantum obfuscator, then coQMA is contained in QSZK.
%\end{theorem}
%\begin{proof}
%(parameters should be checked.) We will actually show coQMA $\subset$ BQP$^{\text{QSZK}}$; since BQP is contained in QSZK, the result will follow. Let $a$ and $b$ satisfy $b-a \leq 1 / \text{poly}(n)$. We will solve $(a, b)$-Identity Check using a subroutine that solves $(\alpha, \beta)$-quantum state distinguishability. 
%
%Let $C$ be the input, i.e., a classical description of an $n$-qubit quantum circuit. Create an identity circuit $D$ with an equal number of inputs as $C$, and of equal length to $C$. Let $O_C$ be a circuit that initializes a register with the classical state $\ket{C}$ containing the classical description of $C$, and applies the circuit of $\mathcal O$ which corresponds to the input length $|C|$. Likewise, let $O_D$ be be a circuit that initializes a register with the classical state $\ket{D}$ containing the classical description of $D$, and applies the circuit of $\mathcal O$ which corresponds to the input length $|D| = |C|$. Note that, after tracing out ancillas, the outputs of these circuits are given by
%$$
%\tr_\text{anc.} \bigl[O_C\ket{0}\bra{0}O_C^\dagger\bigr] = \mathcal O(C)
%\qquad \text{and} \qquad
%\tr_\text{anc.} \bigl[O_D\ket{0}\bra{0}O_D^\dagger\bigr] = \mathcal O(D)\,.
%$$
%Now apply the subroutine for solving quantum state distinguishability to the pair $(O_C, O_D)$. If it says ``close'', we output YES; otherwise we output NO. Let's show that this has solved $(a, b)$-identity-check. Note that the states $\mathcal O(C)$ and $\mathcal O(D)$ must have the same number of qubits, and denote that number by $m$.
%\begin{itemize}
%\item \textbf{completness.} In this case, the obfuscated states satisfy $\|\mathcal O(C) - \mathcal O(D)\|_\text{tr} \leq \alpha$.  By the definition of the induced trace norm, this implies that $\|\mathcal J_{\mathcal O(C)}^n - \mathcal J_{\mathcal O(D)}^n\|_\diamond \leq \alpha$. By functional equivalence for $C$ and $D$ and the triangle inequality, it follows that $\|U_C - U_D\|_\diamond = \|U_C - I\|_\diamond \leq \alpha$, as desired.
%
%\item \textbf{soundness.} In this case, the obfuscated states satisfy $\|\mathcal O(C) - \mathcal O(D)\|_\text{tr} \geq \beta$. We claim that this implies $\|U_C - U_D\|_\diamond > b$. Suppose this is not the case, i.e., that these operators are in fact close; then by the indistinguishability property, it would follow that $\mathcal O(C)$ and $\mathcal O(D)$ are close as well, a contradiction.
%\end{itemize}
%The above amounts to a BQP$^\text{QSZK}$ protocol for a coQMA-hard problem, thus placing coQMA in QSZK.
%\end{proof}

%%%%%%%%%%%%%%%%%%%%%%
\section{Discussion}
%%%%%%%%%%%%%%%%%%%%%%

Open questions:
\begin{itemize}
\item can you achieve single-copy vbb obfuscation with quantum states? What about information-theoretic? For classical or quantum computations?
\item can we extend the proof to show that the state must be consumable? Is that easier in the quantum case?
\item can you achieve quantum circuit-to-circuit obfuscation under the comp. indistinguishability condition?
\item what happens if we think about obfuscating measurements, or CPTP circuits?
\end{itemize}

%%%%%%%%%%%%%%%%%%%%%%%%%%%%%%%%
\bibliography{QuantumCrypto}
%%%%%%%%%%%%%%%%%%%%%%%%%%%%%%%%

%%%%%%%%%%%%%%%%%%%%%%%%%%%%%%%%
\appendix
%%%%%%%%%%%%%%%%%%%%%%%%%%%%%%%%

%%%%%%%%%%%%%%%%%%%%%%%%%%%%%%%%
\section{Old VBB definitions}
%%%%%%%%%%%%%%%%%%%%%%%%%%%%%%%%

We now define the notion of an interpreter, which is simply a quantum algorithm equipped with some additional data.

\begin{definition}
An \textbf{interpreter} is a polynomial-time uniform family of unitary quantum circuits $\mathcal J = \{J_{n, m}\}_{n, m \in \N}$, such that for every $n$ and $m$, $J_{n, m}$ has an $n$-qubit register $A$, an $m$-qubit register $B$, and an ancilla $C$ of size poly$(n, m)$. For every $n \in \N$ and every $m$-qubit state $\rho$, we define a superoperator on $A$ by
$$
\mathcal J_n^\rho : \sigma \mapsto \emph{Tr}_{BC}\bigl( J_{n, m}\bigl[ \sigma \otimes \rho \otimes \ket{0}\bra{0}_C \bigr]J_{n, m}^\dagger\bigr)\,.
$$
\end{definition}

\noindent In applications, we think of $\mathcal J$ as enabling an end-user to apply a superoperator to an input state $\sigma$ with the help of the $m$-qubit ``advice'' state $\rho$, presumably provided by some other party. We say that the state $\rho$ implements the operator $\mathcal J_n^\rho$. A simple example is given by universal circuits: in this case, $\rho$ is a classical description of a quantum circuit for implementing $\mathcal J_n^\rho$, and $J_{n, m}$ consists of a universal sequence of gates which are applied to the first register and controlled by the second register. While this is not explicitly required by the definition, all advice states in this work will be efficiently preparable.


We now wish to consider obfuscated advice. Given a (potentially non-unitary) quantum circuit $C$, let $U_C$ denote the superoperator implemented by $C$. We will frequently refer to ``quantum adversaries'' and ``quantum simulators.'' Both will mean a polynomial-time quantum algorithm which accepts a quantum state as input (along with a polynomial-size initialized ancilla) and outputs a classical bit. We will sometimes make use of quantum simulators which have oracle access to some unitary operator; such a simulator will be denoted by, e.g.,  $\mathcal S^U$. The quantum circuits of such a simulator are allowed to make use of a ``black box'' gate which applies $U$. Each use of a black box counts towards the length of the circuit, which must remain polynomial in the input size.
\begin{definition}
A \textbf{black-box quantum obfuscator} is a pair $(\mathcal J, \mathcal O)$ where $\mathcal J$ is an interpreter and $\mathcal O$ is a probabilistic-quantum algorithm which, on input an $n$-qubit quantum circuit $C$, outputs an $m$-qubit quantum state $\mathcal O(C)$ satisfying
\begin{enumerate}
\item (polynomial slowdown)  $m = \text{poly}(n, |C|)$;
\item (functional equivalence) there exists a negligible $\epsilon_1$ such that $\bigl\| \mathcal J_n^{\mathcal O(C)} - U_C \bigr\|_\diamond \leq \epsilon_1(n, |C|)$;
\item (virtual black-box) for every quantum adversary $\mathcal A$ there exists a quantum simulator $\mathcal S^{U_C}$ and a negligible $\epsilon_2$ such that
$$
\Bigl| \emph{Pr}[\mathcal A(\mathcal O(C)) = 1] - \emph{Pr}\bigl[\mathcal S^{U_C}\bigl(\ket{0}^{\otimes |C|}\bigr) = 1\bigr] \Bigr| \leq \epsilon_2(n, |C|)\,.
$$
\end{enumerate}
\end{definition}
\noindent We will select the functions $\epsilon_1, \epsilon_2$ later, to be the largest possible for which our impossibility proofs still work.

Note that allowing $\mathcal O$ to be probabilistic-quantum allows it to output different states $\mathcal O(C; r)$ depending on the choice of classical randomness $r$. Why is this different from just letting $\mathcal O$ be fully quantum? Suppose that we were to change the above definition to that effect, and consider the case where $C$ is used to flip a single bit. The obfuscator could then, with equal probability, output either a state which always outputs $1$ or a state which always outputs $0$. Because we combined this choice into a density matrix, the functional equivalence condition would technically be satisfied. On the other hand, the actual output state of the algorithm would \emph{never} satisfy functional equivalence -- a frustrating situation for the end-user, to say the least.


%%%%%%%%%%%%%%%%%%%%%%
\section{[OLD NOTES]}
%%%%%%%%%%%%%%%%%%%%%%

\subsection{Preliminaries}
%%%%%%%%%%%%%%%%%%%%%%
Given a probability distribution $X$ on a finite set $S$ and an element $s \in S$, let $s \sim X$ denote the experiment of sampling $s$ according to the distribution $X$. For example, $\Pr_{s \sim X} [s \in S']$ denotes the probability that a sample of $X$ belongs to some subset $S' \subset S$. The total variation distance between two probability distributions $X$ and $Y$ taking values in $S$ is defined by
$$
|X - Y| = \frac{1}{2} \sum_{s \in S} \bigl|\Pr[X=s] - \Pr[Y=s]\bigr|\,.
$$
If $A$ and $B$ are random variables with the same range, the notation $|A-B|$ will mean the total variation distance between the distributions of $A$ and $B$.


We will often refer to circuits as deciding some problem, in the following sense. Let $\{C_n\}_{n \in \N}$ be a uniform family of classical probabilistic circuits. Fix $x \in \{0,1\}^n$, and let $C_n x$ denote the random variable determined by running $C_n$ on the input $x$ (with each remaining input bit set to either $0$ or to the outcome of a uniformly random coinflip) and reading out the value of the first output bit. If the input $x$ is selected according to some probability distribution $A$, then the acceptance probability is $\Pr_{x \sim A}[C_n x = 1]$. It is implicit that the probability is now taken over both the choice of $x$ and the coins of $C_n$.

We can also view quantum circuits in this way. In the remainder of these notes, ``quantum circuit'' will always mean ``unitary quantum circuit.'' Any measurements will be specified explicitly, and performed after the quantum circuit is applied. This is sufficient to describe arbitrary quantum computations (which in general may include many rounds of unitary operations, adapted measurements, and classical pre- and post-processing.) Set $\{C_n\}_{n \in \N}$ to be a uniform family of quantum circuits, and let $p(n)$ denote the number of qubits acted on by $C_n$. Given a quantum state $\ket \psi$ on $n$ qubits, we apply the circuit $C_n$ to the state $\ket \psi \ket{0}^{\otimes p(n)-n}$, and then measure the first qubit in the computational basis. This procedure can be described by a $\{0,1\}$-valued random variable, which we will denote by $M(C_n \ket \psi)$. Specifically, for $a \in \{0,1\}$,
$$
\Pr[M(C_n \ket \psi) = a] = \sum_{x \in \{0,1\}^{p(n)}\,:\,x_1 = a} \left| \bra{x} C_n \ket \psi \ket 0^{\otimes p(n)-n} \right|^2\,.
$$

\begin{itemize}
\item it is worthwhile to discuss here why there is no ``more powerful'' way to use circuit families to solve decision problems (e.g., by appealing to PromiseBQP/PromiseBPP-hardness).
\end{itemize}

A \emph{probability ensemble} $D$ is a sequence $\{D_n\}_{n \in \N}$ of bitstring-valued random variables, such that for some polynomial $\ell$, each $D_n$ takes values in $\{0,1\}^{\ell(n)}$. We may sometimes need such ensembles to be polynomial-time constructible, meaning that one can sample from $D$ via a uniform family of probabilistic circuits. Recall that a function $\phi: \N \rightarrow [0, \infty)$ is \emph{negligible} if it is smaller than inverse-polynomial in $n$.  We identify four distinct notions of indistinguishability of two probability ensembles $A$ and $B$:
\begin{enumerate}
\item \emph{perfectly indistinguishable}: $A_n = B_n$ for all sufficiently large $n$;
\item \emph{statistically indistinguishable}: there exists a negligible function $\phi$ such that $|A_n - B_n| \leq \phi(n)$ for all sufficiently large $n$;
\item \emph{quantumly indistinguishable}: there exists a negligible function $\phi$ such that, given any uniform family $\{C_n\}_{n \in \N}$ of quantum circuits, for all sufficiently large $n$ we have
$$
\left| \Pr_{x \sim A_n} [M(C_n \ket x) = 1] - \Pr_{x \sim B_n} [M(C_n \ket x) = 1] \right| \leq \phi(n)\,.
$$
\item \emph{classically indistinguishable}: there exists a negligible function $\phi$ such that, given any uniform family $\{C_n\}_{n \in \N}$ of classical probabilistic circuits, for all sufficiently large $n$ we have
$$
\left| \Pr_{x \sim A_n} [C_n x = 1] - \Pr_{x \sim B_n}[C_n x = 1] \right| \leq \phi(n)\,;
$$
\end{enumerate}

The quantum case can also be expressed naturally in terms of density operators. Let us write $M(C_n \rho)$ for the random variable corresponding to the same decision experiment as before, but now starting with a density operator $\rho$. Given a probability distribution $A$ on $\{0,1\}^n$, set
$$
\rho_A = \sum_{x \in \{0,1\}^n} \Pr[A = x] \ket x \bra x\,.
$$
The random variable $M(C_n \rho_A)$ then exactly captures the outcome of selecting a string at random according to $A$, and then running the decision experiment corresponding to $C_n$. In other words,
$$
\Pr_{x \sim A} [M(C_n \ket x) = a] = \Pr[M(C_n \rho_A) = a]\,.
$$

\begin{proposition} Indistinguishability of probability ensembles satisfies
$$
\text{perfect} \Rightarrow \text{statistical} \Rightarrow \text{quantum} \Rightarrow \text{classical}\,.
$$
\end{proposition}
\begin{proof}
Let $A$ and $B$ be probability ensembles. The first implication is immediate from the definition. For the second, recall the definition of trace norm $\|\rho\|_1 = \tr \sqrt{\rho \rho^\dagger}$, and note that the trace distance between the two relevant density operators is
$$
\left\| \rho_A - \rho_B \right\|_1 = 2 |A - B|\,.
$$
It's easy to check that the trace norm is unitarily invariant, so applying the same quantum circuit to both $\rho_A$ and $\rho_B$ does not affect the trace distance. The final measurement is just a projection to some subspace, and so the difference in the acceptance probabilities is bounded above by twice the trace distance. For the third implication, by standard arguments we can replace any classical circuit family that distinguishes two ensembles with a classical reversible circuit family that does the same. Reversible circuits are a special case of quantum circuits.
\end{proof}

\begin{itemize}
\item examples of why the implications are strict: (1) trivial, (2) large statistical difference but no quantum distinguisher (graph isomorphism?), and (3) quantum distinguisher but no classical distinguisher (factoring, or Bill's idea?).
\end{itemize}

By the triangle inequality, all four notions of indistinguishability are transitive, i.e. if $A$ is indistinguishable from $B$ and $B$ is indistinguishable from $C$, then $A$ is indistinguishable from $C$.  All four notions of indistinguishability are also closed under applying polynomial-time operations to both ensembles; the exception is that classically indistinguishable ensembles may become classically distinguishable after an efficient quantum algorithm is applied. 



\subsection{Black-box Quantum circuit obfuscation}
%%%%%%%%%%%%%%%%%%%%%%

Given a (not necessarily uniform) family of circuits $\mathcal C$, let $\mathcal C_n$ denote the subset of $\mathcal C$ consisting of all circuits that act on exactly $n$ qubits. If each $\mathcal C_n$ consists of one circuit only, then $C_n$ will refer to that unique circuit, and the expression $\mathcal C \ket x$ will mean $C_n \ket x$ where $n$ is the number of qubits of the state $\ket x$.

For a quantum circuit $C$, let $U_C$ denote the unitary operator implemented by $C$. The notation $S^C$ will stand for a quantum circuit $S$ which, in addition to a universal set of quantum gates, can also make use of an additional black-box gate which implements $U_C$. The black-box gate can be used as many times as needed, although each use does count toward the total length of $S^C$.

We are now ready to define a few different notions of quantum circuit obfuscation. Our definitions closely follow the classical ones in Goldwasser and Rothblum. 

\begin{definition}\label{def:black-box} A classical probabilistic algorithm $\mathcal O$ that takes as input a quantum circuit $C$ and outputs another quantum circuit $\mathcal O(C)$ is a quantum {\bf black-box obfuscator} for the circuit family $\mathcal C$ if it satisfies:
\begin{enumerate}
\item preserving functionality: there is a negligible function $\phi$ such that for any $n$ and any $C \in \mathcal C_n$, 
$$
\Pr[U_C \neq U_{\mathcal O(C)}] \leq \phi(n)\,.
$$
\item polynomial slowdown: there is a polynomial $p$ such that for any $C \in \mathcal C$, $|\mathcal O(C)| \leq p(|C|)$.
\item virtual black-box: For any adversary (uniform quantum circuit family) $\mathcal A$, there is a simulator (uniform quantum circuit family) $\mathcal S$ and a negligible $\phi$ such that
$$
\left| \Pr [M(\mathcal A\ket{\mathcal O(C)}) = 1] - \Pr[M(\mathcal S^C \ket{0}) = 1] \right| \leq \phi(n)
$$
for every $n$ and every $C \in \mathcal C_n$.
\end{enumerate}
\end{definition}

\begin{itemize}
\item in GR there aren't four versions of the last property -- just the computational one. Why?
\item we may later wish to relax the functionality-preserving condition, so that two unitaries are considered functionally equivalent so long as (say) there is no polynomial-length proof of their inequality. This would affect later definitions too.
\item are the classically not-black-box-obfuscatable functions also not quantum black-box obfuscatable?
\item if not, are there other examples of not-quantum-black-box-obfuscatable functions? In order for these examples to be interesting, I guess they shouldn't be ``learnable,'' i.e., you can't figure out exactly what they are with a polynomial number of black-box uses.
\item is there an example family of quantum circuits which *is* black-box obfuscatable?
\end{itemize}

\subsection{Best-possible}
%%%%%%%%%%%%%%%%%%%%%%

In what follows, for the sake of simplicity we omit the perfect, statistical, and classical variants of the definitions; one can arrive at these versions simply by replacing quantum indistinguishability of the relevant ensembles to one of the other notions. We will always be obfuscating quantum circuits, so when the word ``quantum'' appears in front of ``obfuscator'', this refers to the type of indistinguishability. We say that two uniform quantum circuit families $\mathcal C'$ and $\mathcal C''$ are equivalent if they consist of functionally equivalent circuits of the same size; more precisely, for every $n$, $|\mathcal C'_n| = |\mathcal C''_n| = 1$ and $|C'_n| = |C''_n|$  and $U_{C'_n} = U_{C''_n}$.

\begin{itemize}
\item the exact-same-length condition seems too strong, but it does appear in GR too, along with a later comment about how it can be removed. I guess some care is needed.
\end{itemize}

\begin{definition} A classical probabilistic algorithm $\mathcal O$ that takes as input a quantum circuit $C$ and outputs another quantum circuit $\mathcal O(C)$ is a quantum {\bf best-possible obfuscator} for the family $\mathcal C$ if it satisfies properties (1) and (2) from Definition \ref{def:black-box}, as well as the following property:
\begin{enumerate}
\setcounter{enumi}{2}
\item for any learner (uniform quantum circuit family) $\mathcal L$, there is a simulator (uniform quantum circuit family) $\mathcal S$ and a negligible $\phi$ such that, for all uniform equivalent subfamilies $\mathcal C', \mathcal C''$ of $\mathcal C$, the two ensembles $\mathcal L(\mathcal O(\mathcal C'))$ and $\mathcal S(\mathcal C'')$ are quantumly indistinguishable.
\end{enumerate}
\end{definition}

\noindent {\bf Example: quantum witness encryption.} The classical idea of witness encryption is from a paper by Sahai, Garg and others, and the idea of solving it with obfuscation is from the big paper by Sahai et al. In the quantum case, we set up the problem as follows. Suppose Alice wishes to encrypt a quantum plaintext $\ket x$, but not to a particular key or for a particular person; instead, the encryption is tied to a challenge question, and anyone that can answer the question correctly can decrypt the plaintext. Alice outputs a ciphertext $F_\phi \ket x$ where $\phi$ is a quantum 3-SAT formula, such that there exists an efficient algorithm Eval with the property that Eval$(F_\phi\ket x, \ket y) = \ket x$ if $\ket y$ is a satisfying assignment for $\phi$. The security requirement is that if $\phi$ does not have a satisfying assignment, then the ensembles $F_\phi \ket x$ and $F_\phi \ket {x'}$ are quantum indistinguishable (formally, this now requires a definition of distinguishing \emph{quantum} ensembles) whenever $\ket x$ and $\ket{x'}$ are quantum states on the same number of qubits. Note that the definition says nothing about the case where $\phi$ is satisfiable but a satisfying assignment is not known. While this may seem counterintuitive, Sahai and Garg etc. are nonetheless able to construct various interesting encryption schemes (like public-key encryption and identity encryption) from witness encryption. 

The problem of quantum witness encryption can be solved using a quantum best-possible obfuscator $\mathcal O$, as follows. First, Alice selects a random Clifford (or Pauli) circuit $C$. She then writes down a quantum circuit $M_C$ which accepts two registers (and some ancillas), such that $M \ket z \ket y \ket 0 = \ket {C^{-1}z} \ket y \ket 0$ when $\ket y$ is a satisfying assignment for $\phi$, and $M \ket z \ket{y'} \ket 0 = \ket z \ket{y'} \ket 0$ for $\ket{y'}$ not a satisfying assignment for $\phi$. The ciphertext $F_\phi \ket x$ will consist of the pair $(C \ket x, \mathcal O(M_C))$. A recipient with a satisfying assignment $\ket y$ can decrypt by computing $\mathcal O(M_C) \ket {Cx} \ket y \ket 0$. On the other hand, if no satisfying assignment exists, then $M_C$ acts like the identity operator on every input. By the definition of best-possible, a quantum adversary can learn nothing more from $\mathcal O(M_C)$ than she could from the trivial circuit with no gates. Moreover, by the design property of Cliffords (or Paulis) the adversary also observes $\ket{Cx}$ to be a maximally mixed state.

\begin{itemize}
\item Stephen has a description of how to build the circuit $M_C$, and that should be added.
\item I guess the state $C \ket x$ and the circuit $M_C$ are correlated. Is this a problem? This probably has to be addressed by defining quantum indistinguishability of quantum ensembles, and then showing that quantum indistinguishability of the classical ensemble $\mathcal O(M_C)$ plus 2-design property on $C \ket{x}$ implies quantum indistinguishability of the quantum ensemble $(C \ket x, \mathcal O(M_C))$.
\item what does $M_C$ do if you feed in a state that has a little bit of projection into a satisfying assignment? I guess that, unless the size of the projection is $1/$poly, it's still indistinguishable from identity...
\item I have some ideas on why the above is exactly the right definition (e.g., weakening to $\phi$ being just a 3-SAT formula opens it up to being solved by classical obfuscation.) 
\end{itemize}

\subsection{Indistinguishability}
%%%%%%%%%%%%%%%%%%%%%%

\begin{definition} A classical probabilistic algorithm $\mathcal O$ that takes as input a quantum circuit $C$ and outputs another quantum circuit $\mathcal O(C)$ is a quantum {\bf indistinguishability obfuscator} for the family $\mathcal C$ if it satisfies properties (1) and (2) from Definition \ref{def:black-box}, as well as the following property:
\begin{enumerate}
\setcounter{enumi}{2}
\item for all uniform equivalent subfamilies $\mathcal C', \mathcal C''$ of $\mathcal C$, the two ensembles $\mathcal O(\mathcal C')$ and $\mathcal O(\mathcal C'')$ are quantumly indistinguishable.
\end{enumerate}
\end{definition}

\begin{itemize}
\item in all of the above, we could have considered obfuscating quantum states, or even using quantum algorithms to obfuscate classical descriptions of a quantum circuit. Why is this the ``right'' case (or at least an interesting one)?
\end{itemize}

\subsection{Relationships between the definitions}
%%%%%%%%%%%%%%%%%%%%%%

With the definitions set up as above, many of the proofs of Goldwasser and Rothblum go through with little to no changes.

\begin{proposition} There exists an inefficient perfect indistinguishability obfuscator for all quantum circuits.
\end{proposition}
\begin{proof}
The obfuscator just picks the lexicographically first circuit which implements the same unitary as the given circuit. Looping through lexicographically ordered circuits can be done in PSPACE, and equivalence-checking can be done in QMA $\subset$ QIP = PSPACE too.
\end{proof}

\begin{itemize}
\item what's the smallest class that one can do this in?
\end{itemize}

\begin{proposition} If $\mathcal O$ is a best-possible quantum obfuscator for a circuit family $\mathcal C$, then it is also a quantum indistinguishability obfuscator for $\mathcal C$.
\end{proposition}
\begin{proof}
Let $\mathcal C'$ and $\mathcal C''$ be uniform equivalent subfamilies of $\mathcal C$, and let $\mathcal L$ be the trivial learner that simply implements the identity operator. By the best-possible property, there is a simulator $\mathcal S$ such that $\mathcal S(\mathcal C'')$ is quantum indistinguishable from $\mathcal L ( \mathcal O( \mathcal C')) = \mathcal O ( \mathcal C')$. By the same property, we also have that $\mathcal S(\mathcal C'')$ is quantum indistinguishable from $\mathcal L(\mathcal O (\mathcal C'')) = \mathcal O( \mathcal C'')$. By the transitivity property of indistinguishability, it follows that $\mathcal O(\mathcal C')$ is indistinguishable from $\mathcal O(\mathcal C'')$.
\end{proof}

\begin{proposition} If $\mathcal O$ is an efficient quantum indistinguishability obfuscator for a circuit family $\mathcal C$, then it is also an efficient quantum best-possible obfuscator for $\mathcal C$.
\end{proposition}
\begin{proof}
Let $\mathcal C'$ and $\mathcal C''$ be equivalent subfamilies of $\mathcal C$, and let $\mathcal L$ be a (quantum) learner whose output on $\mathcal C'$ is the ensemble $\mathcal L(\mathcal O (\mathcal C'))$. We define a (quantum) simulator by setting $\mathcal S = \mathcal L \circ \mathcal O$; its output on $\mathcal C''$ is then the ensemble $\mathcal L(\mathcal O(\mathcal C''))$. Since the ensembles $\mathcal O(\mathcal C')$ and $\mathcal O(\mathcal C'')$ are quantum indistinguishable, so are their images under $\mathcal L$.
\end{proof}

\subsection{Example: Clifford circuits}
%%%%%%%%%%%%%%%%%%%%%%

Recall that the single-qubit Pauli operators are defined by
$$
I =
\begin{pmatrix} 
1 & 0 \\
0 & 1 \\
\end{pmatrix}\,,
\qquad
X =
\begin{pmatrix} 
0 & 1 \\
1 & 0 \\
\end{pmatrix}\,,
\qquad
Y =
\begin{pmatrix} 
0 & -i \\
i & 0 \\
\end{pmatrix}\,,
\qquad
Z = 
\begin{pmatrix} 
1 & 0 \\
0 & -1 \\
\end{pmatrix}\,.
$$
Each Pauli operator is self-adjoint and unitary. A few useful relations are
$$
X^2 = Y^2 = Z^2 = I
\qquad
XY = -YX = iZ
\qquad
XZ = -ZX = -iY
\qquad 
YZ = -ZY = iX\,.
$$
From these relations, it's easy to see that the set of matrices $\alpha M$ where $\alpha \in \{\pm 1, \pm i\}$ and $M \in \{I, X, Y, Z\}$ forms a group under matrix multiplication. This group is generated by $\{X, Y, Z\}$ and $\{\pm 1, \pm  i\}$. In the $n$-qubit case, we first set
$$
X_j = I^{\otimes j-1} \otimes X \otimes I^{\otimes n-j}
$$
and likewise for $Y_j$ and $Z_j$. We define the $n$-qubit Pauli group $\mathcal P_n$ to be the group generated by $\{X_j, Y_j, Z_j : j = 1, \dots, n \}$  and $\{\pm 1, \pm i\}$.

The Clifford group on $n$ qubits is defined to be the normalizer of the Pauli group inside the unitary group, i.e.,
$$
\mathcal C_n = \{ U \in U(2^n) : U P U^\dagger \in \mathcal P_n \text{ for all } P \in \mathcal P_n\}\,.
$$
By direct computation on the Pauli generators, it's easy to check that the following gates are elements of $\mathcal C_n$ for any $n \geq 2$:
$$
H = \frac{1}{\sqrt{2}}
\begin{pmatrix} 
1 & 1 \\
1 & -1 \\
\end{pmatrix}\,,
\qquad
P = 
\begin{pmatrix} 
1 & 0 \\
0 & i \\
\end{pmatrix}\,,
\qquad
CNOT = 
\begin{pmatrix} 
1 & 0 & 0 & 0 \\
0 & 1 & 0 & 0 \\
0 & 0 & 0 & 1 \\
0 & 0 & 1 & 0
\end{pmatrix}\,.
$$
It is a theorem (see Gottesman's papers) that the above gates (when applied to arbitrary qubits or pairs of qubits) actually generate the entire Clifford group. A Clifford circuit is any circuit which is made up of gates from the above gate set. It is well-known that Clifford circuit computations can be efficiently simulated by a classical computer, but that adding any gate outside the Clifford group yields a quantum-universal set. In spite of their lack of computational power, Clifford circuits are quite relevant in quantum information, e.g., in quantum error correction and quantum cryptography.

In this section, we show how to put any Clifford circuit into a unique normal form. Something like this is already discussed in Gottesman's PI lectures. Selinger also provides a unique normal form (as well as generators and relations for $\mathcal C_n$) but he uses a different gate set. The approach below also seems more natural, as it's closely related to how Cliffords are usually discussed in the QI literature.

For us, a ``unique normal form'' is a map $f$ from Clifford circuits to Clifford circuits, such that \emph{(i.)} $C$ and $f(C)$ always implement the same unitary operator, and \emph{(ii.)} whenever $C_1$ and $C_2$ are circuits which implement the same unitary operator, $f(C_1)$ and $f(C_2)$ are identical as circuits. We will sketch out how this can be done using a polynomial-time classical algorithm. By definition, this immediately gives an indistinguishability obfuscator for Clifford circuits. 

Moreover, by a result of Richard Low, given a black box that implements a Clifford group element $U$, we can ``learn'' the action of $U$ on the Pauli generators in polynomial time. As our algorithm will make clear, knowing the action of $U$ on the generators suffices to produce the normal form. This means that any learner that has access to a normal-form Clifford circuit for $U$ can be simulated by a learner with black-box access to $U$. This obfuscation scheme thus also satisfies the conditions of black-box obfuscation.

Unfortunately, this obfuscation is in some sense trivial; while it is true that the precise form of the initial circuit is not learnable from the obfuscated circuit, it is nonetheless easy to learn the full functionality.

We can map each element of the $n$-qubit Pauli group to a $2n$-bit string by ignoring the phase and setting
$$
X_i \mapsto (\underbrace{0, \dots, 0}_{i-1}, 1, 0, \dots, 0)
\qquad \text{and} \qquad
Z_i \mapsto (\underbrace{0, \dots, 0}_{n+i-1}, 1, 0, \dots, 0)\,.
$$
By checking the relations on the generating set, one sees that this map yields an isomorphism
$$
f: \mathcal P_n / \{\pm 1, \pm I\} \rightarrow \Z_2^{2n}\,.
$$
It's also easy to compute how the conjugation action of a Clifford gate on a Pauli generator affects the corresponding binary string. Since conjugation is linear, this is described by a matrix. For example,
$$
H \mapsto
\begin{pmatrix}
0 & 1 \\
1 & 0 \\
\end{pmatrix}\,,
\qquad
P \mapsto
\begin{pmatrix}
1 & 0 \\
1 & 1 \\
\end{pmatrix}\,,
\qquad
CNOT \mapsto
\begin{pmatrix}
1 & 0 & 0 & 0\\
1 & 1 & 0 & 0\\
0 & 0 & 1 & 1\\
0 & 0 & 0 & 1\\
\end{pmatrix}\,.
$$
In general, for any fixed $n$, applying any of the above gates to a particular qubit (or pair of qubits for CNOT) will correspond to some easily computable $2n \times 2n$ binary matrix. Given a Clifford circuit $C$, we can multiply the matrices corresponding to each gate in $C$ to get a matrix $M(C)$. This matrix satisfies the property
$$
M(C) f(P) = f( M (CPC^\dagger))
$$
for every Pauli $P \in \mathcal P_n$. In fact, it is also the case that $M(C_1) = M(C_2)$ whenever $C_1, C_2$ are two Clifford circuits that implement the same element of the Clifford group. This follows from the isomorphism
$$
\mathcal C_n' \cong \opn{Sp}(2n, \F_2)\,,
$$
where $\mathcal C_n'$ denotes $\mathcal C_n$ modulo $\mathcal P_n$ and arbitrary phases, and $\opn{Sp}(2n, \F_2)$ denotes the group of $2n \times 2n$ symplectic matrices over $\F_2$. Why symplectic? Well, because Clifford elements preserve both commutation and anti-commutation of Pauli group elements, and whether two Pauli group elements commute or anti-commute is captured by a symplectic form of their corresponding binary strings:
$$
PQ = (-1)^{\omega(f(P), f(Q))}QP
$$
where
$$
\omega(x, y) = (x_1, \dots, x_n | y_{n+1}, \dots, y_{2n}) + (y_1, \dots, y_n |  x_{n+1}, \dots, x_{2n})
$$
and $(a | b)$ denotes the dot product modulo $2$.

It now remains to produce a unique Clifford circuit from $M(C)$, and append the right element of $\mathcal P_n.$ The former is done through a row reduction procedure. The key observation is that row reduction operations correspond to left-multiplication by matrices corresponding to gates. Once we have row-reduced $M(C)$ to the identity, we then invert the sequence of gates we applied to output a circuit $C'$. We then know that
$$
C^{-1} C' = P
$$
for some $P \in \mathcal P_n$. By applying each gate of $CC'$ to the Pauli generators, we can compute $P$ and append its inverse to $C'$. This constitutes a unique circuit which is equivalent to $C$ up to overall phases.

\begin{itemize}
\item the above is clearly just a sketch, which we can flesh out if we decide this is really important stuff.
\end{itemize}

\paragraph{Why is this uninteresting}

Note that any canonical form obfuscator is not, in general, a black-box obfuscator. A learner which is given the canonical form of a circuit can, in general, learn something that a learner with only black-box access cannot: namely, the canonical form itself! It's useful here to think about what such an obfuscator does on a family of circuits which are \emph{already} in canonical form.

Now suppose all of the functions computed by the relevant class of circuits are black-box learnable, in the sense that there is an efficient algorithm which can use black-box access to a function $f$ to output a description of any circuit (and hence also the canonical circuit) for computing $f$. Strictly speaking, the canonical-form obfuscator is now also a black-box obfuscator. But now again consider a uniform family of circuits which are already in canonical form. In this case, black-box access can be used to recover the entire original circuit perfectly. This should mean that, in an intuitive sense, obfuscation is completely impossible for this circuit family. This explains why our definitions (as well as the classical ones) are meaningless when we talk about efficiently learnable functions.



\end{document}


